\section{Observaciones directas de todas las variables}

\subsection{Aproximación por esquema de simulación}

Este método no aproxima las densidades de transición directamente 
sino que aproxima la solución de la ecuación diferencial 
estocástica de forma tal que el proceso discretizado tiene 
una densidad de transición que conozcamos 
\cite{iacusSimulationInferenceStochastic2008}. Según el esquema
usado para aproximar la solución, obtenemos una u otra densidad de 
transición. El esquema de Euler-Maruyama, al ser la aproximación 
más básica de la solución, es generalmente el más elegido 
cuando se usa este método de inferencia \cite{ozdemircalikusuFittingItoStochastic2021}.
La aproximación de la solución a la ecuación diferencial estocástica
\begin{equation*}
    dX_t = f_\theta(X_t,t)dt + g_\theta(X_t,t)dW_t
\end{equation*}
donde $X_t\in \mathbb R^d$, $f_\theta:\mathbb R^d \to 
\mathbb R^d$, $g_\theta: \mathbb R^d \to \mathbb R^{d\times q}$ y theta
$W_t\in \mathbb R^q$, por el esquema de Euler-Maruyama está 
dado de forma recursiva por 
\begin{equation}\label{eq:EM_cap311}
    Y_{t_{i+1}}= Y_{t_i} + f_\theta(Y_{t_i},t_i)\Delta t_i + 
g_\theta(Y_{t_i},t_i)(W_{t_{i+1}}-W_{t_i})
\end{equation}
De la Proposición \ref{prop:dist_saltos_MB}, se sabe que 
$W_{t_{i+1}}-W_{t_i}$ tiene distribución $N_q(0,\ \Delta t_i I_q)$ y 
al aplicarle la transformación lineal en \ref{eq:EM_cap311}, 
se sigue que
$$Y_{t_{i+1}}\sim N_d\left(Y_{t_i} + f_\theta(Y_{t_i},t_i)\Delta t_i,\ 
\Delta t_i g_\theta(Y_{t_i},t_i)g_\theta(Y_{t_i},t_i)^T\right)$$
Se tiene entonces que el proceso aproximado tiene densidades de 
transición 
$$p(Y_{t_{i+1}}|Y_{t_i},\theta) = \frac{\exp\left(-\frac{1}{2\Delta_i}(Y_{t_{i+1}}-
Y_{t_i}-f_i\Delta_i)(g_ig_i^T)^{-1}(Y_{t_{i+1}}-
Y_{t_i}-f_i\Delta_i)^T \right)}{\sqrt{2\pi 
\Delta t_i \det(g_ig_i^T)}}$$

donde se han definido $f_i:=f_\theta(Y_{t_i},t_i)$ y 
$g_i:=g_\theta(Y_{t_i},t_i)$.

% Este método de inferencia es compatible para el escenario donde 
% se tiene acceso las observaciones de $X_t$. En el contexto 
% epidemiológico, los datos a los que se tiene acceso en la 
% vida real generalmente son de nuevas infecciones y nuevas 
% recuperaciones (o nuevos fallecimientos). Es decir, sólo 
% tenemos acceso a $Z_{t_i} = X_{t_{i+1}}-X_{t_i}$. De \ref{eq:EM_cap311}
% se sigue que 
% $$Z_i \sim N_d(f(Y_{}))$$

\subsection{Verosimilitud simulada}


