\section{Resultados básicos de probabilidad}

Esta sección se construyó a partir de  
\cite{billingsleyProbabilityMeasure1995,
rinconCursoIntermedioProbabilidad2007,
mardiaMultivariateAnalysis1979}. Se asume que el 
lector tiene ya conocimento de definiciones básicas de 
la teoría de probabilidad.
Considerese un espacio de probabilidad $(\Omega,
\mathcal F, P)$ definido como en 
\cite{rinconCursoIntermedioProbabilidad2007}. 

% ----------------------------------------- %
% NUEVA SUBSECCIÓN 
% ----------------------------------------- %

\subsection{Variables y vectores aleatorios} 

Para una definción de variable y vector aletorios se puede 
consultar a \cite{rinconCursoIntermedioProbabilidad2007}. 
Para una introducción a la teoría de medida, véase 
\cite{billingsleyProbabilityMeasure1995}.

\begin{definition}
    Se dice que una función $g:\mathbb R^n\to \mathbb R^m$  
    es Borel medible si $g^{-1}(B)\in \mathcal B(\mathbb R^n)$ para 
    cada $B\in \mathcal B(\mathbb R^m)$.
\end{definition}

\begin{proposition}
    Si $f:\mathbb R^n\to \mathbb R^m$ es continua, entonces es
    Borel medible.
\end{proposition}

\begin{proof}
    Véase \cite{billingsleyProbabilityMeasure1995}.
\end{proof}

\begin{theorem}[Teorema de cambio de variable]
    Sea $X$ una varible aleatoria continua con soporte en 
    $(a,b)\subseteq \mathbb R$ y función de densidad $f_X$. Sea 
    $g:(a,b)\to \mathbb R$ una función continua, estrictamente 
    creciente o decrediente y cuya inversa es diferenciable. 
    Entonces $Y=g(X)$ toma valores en el intervalo $g(a,b)$ y 
    su función de densidad es 
    $$f_Y(y) = 
    \begin{cases}
    f_X(g^{-1}(y))\left|\frac{d}{dy}g^{-1}(y)\right| & \text{para } y\in g(a,b)\\
    0 & \text{en otro caso}
    \end{cases}$$
\end{theorem}

\begin{proof}
    Véase \cite{rinconCursoIntermedioProbabilidad2007}.
\end{proof}

\noindent\textbf{Ejemplo.} Sea $Y=\frac{1}{X}$ donde $X\sim U(a,b)$, 
con $0<a<b$. La función $g(x) = \frac{1}{x}$ es continua y 
estrictamente decreciente en $(a,b)$ y su inversa, que es ella
misma, es diferenciable. Entonces 
$$f_Y(y) = \frac{1}{(b-a)y^2}
\cdot \mathbb I_{\left(\frac{1}{b},\frac{1}{a}\right)}(y)$$

\subsection{Independencia}

\begin{definition}
    Los vectores aleatorios $X=(X_1,...,X_n)$ y $Y=(Y_1,...,Y_m)$
    son independientes, si para cada $A\in \mathcal B(\mathbb R^n)$
    y cada $B\in \mathcal B(\mathbb R^m)$ se cumple que
    $$P(X\in A,\ Y\in B)=P(X\in A)P(X\in B)$$
\end{definition}

\begin{proposition}\label{prop:fun_vect_indep}
    Sean $X=(X_1,...,X_n)$ y $Y=(Y_1,...,Y_p)$ independientes. Si 
    $f:\mathbb R^n \to \mathbb R^m$ y $g:\mathbb R^p \to \mathbb R^q$
    son dos funciones Borel medibles, entonces los vectores
    aleatorios $f(X)$ y $g(Y)$ son independientes.
\end{proposition}

\begin{proof}
    Sean $A$ y $B$ cualesquiera dos conjuntos de Borel en $\mathbb R^n$
    y $\mathbb R^m$, repectivamente. Entonces
    \begin{align*}
        P(f(X)\in A,\ g(X)\in B)&= P(X\in f^{-1}(A),Y\in f^{-1}(B))\\
        &=  P(X\in f^{-1}(A))P(Y\in f^{-1}(B))\\
        &=  P(f(X)\in A)P(g(Y)\in B)
    \end{align*}
\end{proof}

\subsection{Distribuciones}

Prueba 2

\begin{definition}[Distribución Normal Multivariada]
    Se dice que el vector aleatorio $X$ tiene una distribución
    Normal $p$-variada (multivariada si no se hace referencia 
    a la dimensión de $X$) si y solo si $v^TX$ es Normal univaridada
    para todo $v\in \mathbb R^p$.
\end{definition}

Además, si $X$ de distribuye Normal $p$-variada y tiene media 
$\mu\in \mathbb R^p$ y matriz de covarianza definida positiva
$\Sigma \in \mathbb R^{p\times p}$, entonces se escribe
$X\sim N_p(\mu,\ \Sigma)$ y su función de densidad es 
\begin{equation*}
    f(x) = \frac{1}{\sqrt{2\pi\det(\Sigma)}}\cdot 
    \exp\left(
        -\frac{1}{2}(x-\mu)^T \Sigma^{-1} (x-\mu)
    \right)
\end{equation*}

\begin{theorem}
Si $X\sim N_p(\mu,\ \Sigma)$ y $Y=AX+c$, donde 
$A\in \mathbb R^{p\times q}$ y $c\in \mathbb R^q$, 
entonces $Y\sim N_q(A\mu + c,\ A\Sigma A^T)$.
\end{theorem}

Para más propiedades útiles de la distribución Normal multivarida puede 
consultarse a \cite{mardiaMultivariateAnalysis1979}.