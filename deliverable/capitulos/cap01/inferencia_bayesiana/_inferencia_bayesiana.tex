\section{Inferencia bayesiana}

Supongamos que tenemos una muestra $\mathcal D_N = \left(y_1,...,y_N\right)$ de variables aleatorias independientes e idénticamente distribuidas 
cuya función de densidad $f$ está parametrizada por $\boldsymbol{\theta} = (\theta_1,...,\theta_K) \in \Theta$, cuyo valor real es desconocido. 
La \textit{inferencia (estadística)} toma las observaciones $\mathcal D$, las procesa y trata de encontrar formas de describir a $\boldsymbol{\theta}$. \\

En el \textit{enfoque Bayesiano} de la inferencia estadística suponemos que la probabilidad es una medida de la incertidumbre, que resulta útil para analizar fenómenos que no podemos reproducir tantas veces como queramos. En este contexto, se trata a $\btheta$ como una variable aleatoria (o vector aleatorio, según el valor de $K$) y las observaciones que recolectemos nos ayudarán a ajustar nuestras creencias acerca de $\btheta$.\\

El análisis Bayesiano sigue este proceso: 

\begin{enumerate}
    \item Toda la información y creencias que tenemos acerca de $\btheta$ se cuantifican dentro de la \textit{distribución a priori}, $p(\btheta)$.
    \item Se recolecta una muestra $\mathcal D_n$, cuya generación suponemos que depende del parámetro. Debemos entonces relacionar las observaciones y el parámetro a través de la función de \textit{verosimilitud}, $p(\mathcal D_n|\btheta)$.
    \item Ajustamos la información que tenemos de $\btheta$ dado que hemos observado la muestra. Esto se hace a través del Teorema de Bayes,
    $$p(\btheta|\mathcal D_n) = \frac{p(\mathcal D_n|\btheta)p(\btheta)}{p(\mathcal D_n)}.$$
    La inferencia se hace sobre la llamada \textit{distribución posteriori}, $p(\btheta|\mathcal D_n)$.
\end{enumerate}

\subsection{Monte Carlo: clásico y por cadenas de Markov}

% desarrollar en qué casos es difícil de calcular
% hablar de que es para modelos que valen la pena
En los modelos bayesianos, la integral que normaliza la distribución a posteriori 
es, en general, es difícil de calcular y por ello los métodos frecuentistas eran preferidos. 
Sin embargo, el desarrollo de poder de cómputo accesible junto con la teoría de los métodos de Monte Carlo 
que surgió a mediados del siglo XX revitalizaron el interés por la estadística Bayesiana. 

\subsubsection{Monte Carlo clásico}

Esta presentación de los métodos de Monte Carlo ha sido tomada de \cite{doucetSequentialMonteCarlo2001}, pues 
sirve el mismo propósito: presentar los filtros de partículas. Dada una variable aleatoria con función de 
distribución $f$ condicionada, el valor 
esperado de una función de tal variable está dado por

\begin{equation}\label{eq:esperanza_MC}
    I(h) = \mathbb E\left[h(x)|y\right]=\int h(x)f(x|y)dx.
\end{equation} 

Esta integral sólo puede ser evaluada analíticamente en contados casos, por lo que es 
necesario utilizar métodos de aproximación, siendo los métodos de Monte Carlo uno de 
los más usados dado su facilidad de implementación e invarianza respecto a la 
dimensionalidad de la integral.

Si se tiene la capacidad de obtener $N$ muestras independientes e idénticamente distribuidas $\{x_{(i)},i=1,...,N\}$
de $f(x|y)$, entonces se puede contruir el estimador empírico de \ref{eq:esperanza_MC} con
\begin{equation}
    I_N(h) = \frac{1}{N}\sum_{i=1}^{N}h(x^{(i)}),
\end{equation}
que resulta ser un estimador insesgado. Por la ley fuerte de los grandes números, 
\begin{equation}
    I_N(h) \to I(h)
\end{equation}
casi seguramente cuando $N\to \infty$. 

\subsubsection{Algoritmo de Metropolis-Hastings}

Dada una densidad objetivo $f$, el algoritmo de Metropolis-Hastings contruye una 
cadena de Markov con distribución límite $f$, es decir que, a la larga, al tomar 
muestras de esta cadena se está muestreando de una aproximación a la verdadera
distribución objetivo. Para implementar esta aproximación se necesita de una 
función de transición $q:S\times S\to \mathbb R$. La construcción de esta 
cadena se presenta en el algoritmo \ref{alg:Metropolis-Hastings}. La validez de este 
algoritmo ha sido ya demostrada, véase \cite{rinconElementosSimulacionEstocastica2024}. 

\begin{algorithm}
\caption{Muestreo de Metropolis-Hastings}
\label{alg:Metropolis-Hastings}
\begin{algorithmic}[1]
\vspace{0.2cm}
\Require \parbox[t]{13cm}{
    Función de densidad objetivo $f$, función de transición $q(x,y)$,
    estado inicial $x_0\in S$
}
\vspace{0.2cm}
\Ensure 
\parbox[t]{13cm}{
    Muestra de una cadena de Markov con distribución estacionaria $f$
}
\vspace{0.2cm}
\State $x \leftarrow x_0$
\For{$t\in 1,2,...$}
    \State $y \sim q(x, y)$
    \State $\alpha = \min\left\{\frac{f(y)}{f(x)}\cdot \frac{q(y,x)}{q(x,y)},\ 1\right\}$
    \State $u \sim U(0,1)$
    \If{$u \leq \alpha$}
        \State $x \leftarrow y$
    \EndIf
\EndFor
\end{algorithmic}
\end{algorithm}





\subsection{Modelos de Markov parcialmente observados}


Sea $\btheta\in\Theta\subseteq \mathbb R^K$. Para cada valor fijo de $\btheta$, el proceso estocástico $\boldsymbol{X} = \{X(t,\btheta)\}_{t\in T}$ 
representa un sistema dinámico que no podemos observar directamente y cuya evolución futura depende únicamente del valor actual del sistema, 
\begin{equation}\label{eq:MarkovProperty}
    p(X_{\tau_{n+1}}|\ X_{\tau_0},...,X_{\tau_n},\btheta) = f(X_{\tau_{n+1}}|X_{\tau_n}, \btheta)
\end{equation}
es decir, $\boldsymbol{X}$ es un proceso de Markov, para $\tau_0<...<\tau_n<\tau_{n+1}$. En favor de la claridad, se utilizá la siguiente notación $X_i$ para denotar a  $X(t_i, \btheta)$ cuando sea claro que $\btheta$ está fijo 
y $X_{i:j}$ para aludir a $(X_i,\ X_{i+1},\ ...,\ X_j)$.\\

El proceso $\boldsymbol{X}$ sólo puede ser observado indirectamente y es través de otro proceso $\boldsymbol{Y} = \{Y(t, \btheta)\}_{t\in T_{obs}}$ donde $T_{obs}=\{t_i\in 
T,\ i=1,...,N\}$ son los tiempos en los que se realizan estas observaciones. Sea $t_0\in T$ el tiempo en que inicia el proceso $\boldsymbol{X}$ y $t_0\leq t_1 < t_2 <\ ...\ <t_{N}$. En general, el conjunto de índices de las observaciones es subconjunto de los Naturales. Las variables aleatorias observables $Y_{1:N}$ son condicionalmente independientes dadas $X_{0:N}$ y la observación al tiempo actual sólo depende del estado del sistema en tal momento,
\begin{equation}\label{eq:ObservationProbability}
    p(Y_{n}|\ Y_{1:n-1}, X_{0:n}, \btheta) = g(Y_n| X_n, \btheta) 
\end{equation}
Finalmente, al tiempo $t_0$, el estado inicial del sistema está sujeto a la distribución inicial
\begin{equation}\label{eq:InitialDistribution}
    p(X_{0}|\btheta) = \mu(\btheta)
\end{equation}

\begin{definition}
    Un sistema dinámico $X$ con conjunto de observaciones $Y$ que cumple con \ref{eq:MarkovProperty}, 
    \ref{eq:ObservationProbability} y \ref{eq:InitialDistribution} es un 
    \textit{proceso de Markov parcialmente observado}, denotado como 
    \begin{align}\label{eq:POMP}
       \begin{split}
        X_{n}&\sim f(X_{n-1},\btheta)\\
        Y_{n}&\sim g(X_n,\btheta)\\
        X_0&\sim\mu(\btheta)
    \end{split} 
    \end{align}
\end{definition}

En la literatura, a \ref{eq:POMP} también se le conoce como \textit{state-space model} 
o \textit{hidden Markov model}. Algunas de las aplicaciones de esta clase de modelos 
pueden ser consulatadas en \cite{sarkkaBayesianFilteringSmoothing2023}, estas 
abarcan los campos de la navegación, la ingeniería, telecomunicaciones, física y 
otros más.

Dado un modelo de Markov parcialmente observado (eq. \ref{eq:POMP}), se ha desarrollado
suficiente maquinaria teórica y algorítmica para realizar las siguientes tareas 
\begin{enumerate}
    \item El \textit{filtrado} consiste en obtener la distribución marginal del 
    estado al tiempo actual usando la observación actual y aquellas 
    obtenidas con anterioridad, es decir, se calcula 
    $$p(X_k|Y_{1:k}),\quad k = 1,...,N$$
    \item El \textit{suavizado} consiste en obtener la distribución marginal 
    del estado en un tiempo anterior usando todas las observaciones acumuladas hasta 
    el tiempo actual, es decir, se calcula 
    $$p(X_k|Y_{1:N}),\quad j = 1,...,N-1$$
    \item Para la \textit{predicción} se calcula la distribución marginal del estado 
    en algún tiempo futuro, es decir, se obtiene 
    $$p(X_{k+n}|Y_{1:k}),\quad k = 1,...,N,\quad n=1,2,...$$
\end{enumerate}

En este trabajo, se centra la atención en los algoritmos de filtrado 
pues estos son utilizados posteriormente para construir métodos de inferencia 
en el contexto de observaciones indirectas, 
como el \textit{particle Markov chain Monte Carlo} que será introducido 
en el capítulo 3. Las tareas de suavizado y predicción quedan fuera del 
alcance de este trabajo, pero se refiere a \cite{sarkkaBayesianFilteringSmoothing2023}
para una introducción.\\

Los algoritmos de filtrado de partículas son aproximaciones de Monte Carlo al problema 
del filtrado. Fueron desarrollados para atacar problemas no lineales para los cuales 
otros algoritmos populares, como el filtrado de Kalman, no están diseñados. En particular, 
el algoritmo Bootsrap filter (algoritmo \ref{alg:BF}) es la versión básica de los algoritmos 
de fitlrado de partículas y es suficiente para las necesidades de este trabajo. 
Para conocer más algoritmos de esta clase, y su sustentación teórica, se refiere a 
\cite{sarkkaBayesianFilteringSmoothing2023}.

\begin{algorithm}
\caption{Bootstrap filter}
\label{alg:BF}
\begin{algorithmic}[1]
\vspace{0.2cm}
\Require \parbox[t]{13cm}{
    Distribución de transición $f$, distribución de medida $g$, distribución 
    inicial $\mu$, parámetros $\theta$, observaciones $Y_{1:T}$, número 
    de partículas $N$
}
\vspace{0.2cm}
\Ensure 
\parbox[t]{13cm}{
    Estimaciones del valor esperado de los estados $X_{1:T}$
}
\vspace{0.2cm}
\State $x_0^{(i)} \sim \mu(x,\theta)$, $i=1,...,N$
\For{$t\in 1,2,...,T$}
    \State $x_t^{(i)}\sim f(x_t|x_{t-1}^{(i)})$
    \State $w_{t}^{i} \propto g(y_t|x_t^{(i)})$ 
    \State $x_t^{(i)} \sim resampling(x_t,w_t)$
    \State $\mathbb E\left(h(x_t)\right) \approx \sum_{i=1}^N w_t^{i}\cdot h(x_t^{(i)})$
\EndFor
\end{algorithmic}
\end{algorithm}

Nótese que en el algoritmo $\ref{alg:BF}$, el símbolo $\propto$ implica dos tareas, a saber,
la asignación del valor y la posterior normalización. Además, el método de remuestreo varía
según distintas implementaciones. 


