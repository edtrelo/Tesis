\section{Procesos estocásticos}

\cite{rinconIntroduccionProcesosEstocasticos2012}

\begin{definition}\label{def:proceso_estocastico}
    Un proceso estocástico es una colección de variables
    aleatorias $X = \{X_t,\ t\in I\}$ parametrizadas por el
    conjunto $T$, donde todas las variables toman valores
    dentro del conjunto $S$ llamado espacio de estados. 
\end{definition}

% Además, dependiendo de la naturaleza de $I$ y de $S$, los 
% procesos estocásticos reciben distintos nombres, como se
% muestra en la tabla \ref{tab:procesos_estocasticos}

% \begin{table}[h]
%     \centering
%     \begin{tabular}{c c c}
%             & $I$ contable & $I$ incontable \\
%     $S$ discreto   & discreto a tiempo discreto            &  \\
%     $S$ continuo   &              &  
%     \end{tabular}
%     \caption{Una tabla sencilla}
%     \label{tab:procesos_estocasticos}
% \end{table}

\begin{definition}
    Sea $X = \{X_t,\ t\in I\}$ un proceso estocástico. Si 
    para cualesquiera tiempos $0\leq t_1<t_2<...<t_n$
    en $I$ se cumple que 
    $$p(x_{t_n}|x_{t_{n-1},...,t_{t_1}})=p(x_{t_n}|x_{t_{n-1}})$$
    entonces se dice que $X$ tiene la propiedad de Markov.
\end{definition}

\begin{definition}
    Dos procesos estocásticos $\{X_t\}_{t\geq 0}$ y 
    $\{Y_t\}_{t\geq 0}$
\end{definition}

\subsection{Movimiento Browniano}

\begin{definition}
    Un movimiento Browniano (unidimensional) estándar es un proceso 
    estocástico $\{W_t\}_{t\geq0}$ que cumple con las siguientes 
    propiedadaes:
    \begin{enumerate}
        \item $W_0=0$, c.s.
        \item Las trayectorias son continuas.
        \item El proceso tiene incrementos independientes. 
        \item Para cualesquiera $0\leq s<t$, $W_t-W_s\sim N(0,t-s)$
    \end{enumerate}
\end{definition}

\begin{definition}
    Un proceso estocástico $d$-dimensional $W = \{W_t^1,...,
    W_t^d\}_{t\geq 0}$ es un movimiento Browniano estándar
    $d$-dimensional si $\{W_t^i\}$ es un movimiento Browniano 
    estándar para cada $i=1,...,d$ y los procesos $\{W_t^1\},...,\{W_t^d\}$ 
    son independientes dos a dos.
\end{definition}

\begin{proposition}\label{prop:dist_saltos_MB}
    Sea $W$ un movimiento Browniano estándar $d-$dimensional. Entonces 
    para $0\leq s < t$ se tiene que 
    $$W_t-W_s \sim N_d(0,\ (t-s)I_d)$$
\end{proposition}

\begin{proof}
Considerese a $X=(W_t^i,\ W_s^i)$ y $Y=(W_t^j,\ W_s^j)$, para
$i,j=1,...,d$ con $i\not = j$. Sea $g(x,y)=x-y$, que por su 
continuidad es Borel medible. Dado que los procesos son independientes, 
entonces $X$ y $Y$ son independientes y por la proposición \ref{prop:fun_vect_indep}
se tiene que $g(X)=W_t^i-W_s^i$ y $g(Y)=W_t^j-W_s^j$ son independientes.

Luego, para cada $i=1,...,d$, $W_t^i-W_s^i\sim N(0, t-s)$. Por la 
independencia dos a dos se tiene que $W_t-W_s \sim N_d(0,\ (t-s)I_d)$
\end{proof}

