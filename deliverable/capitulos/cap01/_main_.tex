En este capítulo se hace un breve repaso de las definiciones
y resultados necesarios para el uso de ecuaciones diferenciales 
estocásticas (EDSs) en el modelaje de fenómenos físicos. Esto 
incluye los fundamentos teóricos para la construcción de las EDEs 
y las herramientas de inferencia estadística que serán de 
utilidad para el ajuste de modelos contruidos con EDEs a datos 
reales. Los conceptos de probabilidad que se decidieron incluir 
están inspirados en los capítulos introductorios de 
\textcite{
arnoldStochasticDifferentialEquations1974,
oksendalStochasticDifferentialEquations2003,
maoStochasticDifferentialEquations2011,
rinconIntroduccionProcesosEstocasticos2012, 
dobrowIntroductionStochasticProcesses2016}. 
%La parte estadística corresponde a la que usan un subconjunto
%de métodos de inferencia presentados en 
%\textcite{
%iacusSimulationInferenceStochastic2008, 
%fuchsInferenceDiffusionProcesses2013}.

\section{Resultados básicos de probabilidad}

Esta sección se construyó a partir de  
\cite{billingsleyProbabilityMeasure1995,
rinconCursoIntermedioProbabilidad2007,
mardiaMultivariateAnalysis1979}. Se asume que el 
lector tiene ya conocimento de definiciones básicas de 
la teoría de probabilidad.
Considerese un espacio de probabilidad $(\Omega,
\mathcal F, P)$ definido como en 
\cite{rinconCursoIntermedioProbabilidad2007}. 

% ----------------------------------------- %
% NUEVA SUBSECCIÓN 
% ----------------------------------------- %

\subsection{Variables y vectores aleatorios} 

Para una definción de variable y vector aletorios se puede 
consultar a \cite{rinconCursoIntermedioProbabilidad2007}. 
Para una introducción a la teoría de medida, véase 
\cite{billingsleyProbabilityMeasure1995}.

\begin{definition}
    Se dice que una función $g:\mathbb R^n\to \mathbb R^m$  
    es Borel medible si $g^{-1}(B)\in \mathcal B(\mathbb R^n)$ para 
    cada $B\in \mathcal B(\mathbb R^m)$.
\end{definition}

\begin{proposition}
    Si $f:\mathbb R^n\to \mathbb R^m$ es continua, entonces es
    Borel medible.
\end{proposition}

\begin{proof}
    Véase \cite{billingsleyProbabilityMeasure1995}.
\end{proof}

\begin{theorem}[Teorema de cambio de variable]
    Sea $X$ una varible aleatoria continua con soporte en 
    $(a,b)\subseteq \mathbb R$ y función de densidad $f_X$. Sea 
    $g:(a,b)\to \mathbb R$ una función continua, estrictamente 
    creciente o decrediente y cuya inversa es diferenciable. 
    Entonces $Y=g(X)$ toma valores en el intervalo $g(a,b)$ y 
    su función de densidad es 
    $$f_Y(y) = 
    \begin{cases}
    f_X(g^{-1}(y))\left|\frac{d}{dy}g^{-1}(y)\right| & \text{para } y\in g(a,b)\\
    0 & \text{en otro caso}
    \end{cases}$$
\end{theorem}

\begin{proof}
    Véase \cite{rinconCursoIntermedioProbabilidad2007}.
\end{proof}

\noindent\textbf{Ejemplo.} Sea $Y=\frac{1}{X}$ donde $X\sim U(a,b)$, 
con $0<a<b$. La función $g(x) = \frac{1}{x}$ es continua y 
estrictamente decreciente en $(a,b)$ y su inversa, que es ella
misma, es diferenciable. Entonces 
$$f_Y(y) = \frac{1}{(b-a)y^2}
\cdot \mathbb I_{\left(\frac{1}{b},\frac{1}{a}\right)}(y)$$

\subsection{Independencia}

\begin{definition}
    Los vectores aleatorios $X=(X_1,...,X_n)$ y $Y=(Y_1,...,Y_m)$
    son independientes, si para cada $A\in \mathcal B(\mathbb R^n)$
    y cada $B\in \mathcal B(\mathbb R^m)$ se cumple que
    $$P(X\in A,\ Y\in B)=P(X\in A)P(X\in B)$$
\end{definition}

\begin{proposition}\label{prop:fun_vect_indep}
    Sean $X=(X_1,...,X_n)$ y $Y=(Y_1,...,Y_p)$ independientes. Si 
    $f:\mathbb R^n \to \mathbb R^m$ y $g:\mathbb R^p \to \mathbb R^q$
    son dos funciones Borel medibles, entonces los vectores
    aleatorios $f(X)$ y $g(Y)$ son independientes.
\end{proposition}

\begin{proof}
    Sean $A$ y $B$ cualesquiera dos conjuntos de Borel en $\mathbb R^n$
    y $\mathbb R^m$, repectivamente. Entonces
    \begin{align*}
        P(f(X)\in A,\ g(X)\in B)&= P(X\in f^{-1}(A),Y\in f^{-1}(B))\\
        &=  P(X\in f^{-1}(A))P(Y\in f^{-1}(B))\\
        &=  P(f(X)\in A)P(g(Y)\in B)
    \end{align*}
\end{proof}

\subsection{Distribuciones}

Prueba 2

\begin{definition}[Distribución Normal Multivariada]
    Se dice que el vector aleatorio $X$ tiene una distribución
    Normal $p$-variada (multivariada si no se hace referencia 
    a la dimensión de $X$) si y solo si $v^TX$ es Normal univaridada
    para todo $v\in \mathbb R^p$.
\end{definition}

Además, si $X$ de distribuye Normal $p$-variada y tiene media 
$\mu\in \mathbb R^p$ y matriz de covarianza definida positiva
$\Sigma \in \mathbb R^{p\times p}$, entonces se escribe
$X\sim N_p(\mu,\ \Sigma)$ y su función de densidad es 
\begin{equation*}
    f(x) = \frac{1}{\sqrt{2\pi\det(\Sigma)}}\cdot 
    \exp\left(
        -\frac{1}{2}(x-\mu)^T \Sigma^{-1} (x-\mu)
    \right)
\end{equation*}

\begin{theorem}
Si $X\sim N_p(\mu,\ \Sigma)$ y $Y=AX+c$, donde 
$A\in \mathbb R^{p\times q}$ y $c\in \mathbb R^q$, 
entonces $Y\sim N_q(A\mu + c,\ A\Sigma A^T)$.
\end{theorem}

Para más propiedades útiles de la distribución Normal multivarida puede 
consultarse a \cite{mardiaMultivariateAnalysis1979}.
\section{Procesos estocásticos}

Esta sección reintroduce algunos resultados escenciales acerca de 
los procesos estocásticos necesarios para la construcción de la integral
estocástica. Se ha elaborado a partir de 
\cite{rinconIntroduccionProcesosEstocasticos2012,dobrowIntroductionStochasticProcesses2016}.

\begin{definition}\label{def:proceso_estocastico}
    Un proceso estocástico es una colección de variables
    aleatorias $X = \{X_t,\ t\in I\}$ parametrizadas por el
    conjunto $T$, donde todas las variables toman valores
    dentro del conjunto $S$ llamado espacio de estados. 
\end{definition}

\begin{definition}
    Dos procesos estocásticos $\{X_t\}_{t\geq 0}$ y 
    $\{Y_t\}_{t\geq 0}$ son independientes si para cualesquiera dos enteros 
    $n$ y $m$, y tiempos $0\leq t_1<t_2<...<t_n$ y $0\leq s_1<s_2<...<s_m$, 
    se cumple que 
    $$F_{X_{t_1},...,X_{t_n},Y_{s_1},...,Y_{s_m}}(x_1,...,x_n,y_1,...,y_m)=F_{X_{t_1},...,X_{t_n}}(x_1,...,x_n)F_{Y_{s_1},...,Y_{s_m}}(y_1,...,y_m).$$
\end{definition}

\subsection{Cadenas de Markov}

\begin{definition}
    Sea $X = \{X_t,\ t\in I\}$ un proceso estocástico. Si 
    para cualesquiera tiempos $0\leq t_1<t_2<...<t_n$
    en $I$ se cumple que 
    $$p(x_{t_n}|x_{t_{n-1},...,t_{t_1}})=p(x_{t_n}|x_{t_{n-1}}),$$
    entonces se dice que $X$ tiene la propiedad de Markov.
\end{definition}

\subsection{Movimiento Browniano}

\begin{definition}
    Un movimiento Browniano (unidimensional) estándar es un proceso 
    estocástico $\{W_t\}_{t\geq0}$ que cumple con las siguientes 
    propiedadaes:
    \begin{enumerate}
        \item $W_0=0$, c.s.
        \item Las trayectorias son continuas.
        \item El proceso tiene incrementos independientes. 
        \item Para cualesquiera $0\leq s<t$, $W_t-W_s\sim N(0,t-s)$.
    \end{enumerate}
\end{definition}

\begin{definition}
    Un proceso estocástico $d$-dimensional $W = \{W_t^1,...,
    W_t^d\}_{t\geq 0}$ es un movimiento Browniano estándar
    $d$-dimensional si $\{W_t^i\}$ es un movimiento Browniano 
    estándar para cada $i=1,...,d$ y los procesos $\{W_t^1\},...,\{W_t^d\}$
    son independientes dos a dos.
\end{definition}

\begin{proposition}\label{prop:dist_saltos_MB}
    Sea $W$ un movimiento Browniano estándar $d-$dimensional. Entonces 
    para $0\leq s < t$ se tiene que 
    $$W_t-W_s \sim N_d(0,\ (t-s)I_d).$$
\end{proposition}

\begin{proof}
Considerese a $X=(W_t^i,\ W_s^i)$ y $Y=(W_t^j,\ W_s^j)$, para
$i,j=1,...,d$ con $i\not = j$. Sea $g(x,y)=x-y$, que por su 
continuidad es Borel medible. Dado que los procesos son independientes, 
entonces $X$ y $Y$ son independientes y por la proposición \ref{prop:fun_vect_indep}
se tiene que $g(X)=W_t^i-W_s^i$ y $g(Y)=W_t^j-W_s^j$ son independientes.

Luego, para cada $i=1,...,d$, $W_t^i-W_s^i\sim N(0, t-s)$. Por la 
independencia dos a dos se tiene que $W_t-W_s \sim N_d(0,\ (t-s)I_d)$.
\end{proof}


\section{Inferencia frecuentista}

\subsection{Estimación por Máxima verosimilitud}

\section{Inferencia bayesiana}

Supongamos que tenemos una muestra $\mathcal D_N = \left(y_1,...,y_N\right)$ de variables aleatorias independientes e idénticamente distribuidas 
cuya función de densidad $f$ está parametrizada por $\boldsymbol{\theta} = (\theta_1,...,\theta_K) \in \Theta$, cuyo valor real es desconocido. 
La \textit{inferencia (estadística)} toma las observaciones $\mathcal D$, las procesa y trata de encontrar formas de describir a $\boldsymbol{\theta}$. \\

En el \textit{enfoque Bayesiano} de la inferencia estadística suponemos que la probabilidad es una medida de la incertidumbre, que resulta útil para analizar fenómenos que no podemos reproducir tantas veces como queramos. En este contexto, se trata a $\btheta$ como una variable aleatoria (o vector aleatorio, según el valor de $K$) y las observaciones que recolectemos nos ayudarán a ajustar nuestras creencias acerca de $\btheta$.\\

El análisis Bayesiano sigue este proceso: 

\begin{enumerate}
    \item Toda la información y creencias que tenemos acerca de $\btheta$ se cuantifican dentro de la \textit{distribución a priori}, $p(\btheta)$.
    \item Se recolecta una muestra $\mathcal D_n$, cuya generación suponemos que depende del parámetro. Debemos entonces relacionar las observaciones y el parámetro a través de la función de \textit{verosimilitud}, $p(\mathcal D_n|\btheta)$.
    \item Ajustamos la información que tenemos de $\btheta$ dado que hemos observado la muestra. Esto se hace a través del Teorema de Bayes,
    $$p(\btheta|\mathcal D_n) = \frac{p(\mathcal D_n|\btheta)p(\btheta)}{p(\mathcal D_n)}.$$
    La inferencia se hace sobre la llamada \textit{distribución posteriori}, $p(\btheta|\mathcal D_n)$.
\end{enumerate}

\subsection{Monte Carlo: clásico y por cadenas de Markov}

% desarrollar en qué casos es difícil de calcular
% hablar de que es para modelos que valen la pena
En los modelos bayesianos, la integral que normaliza la distribución a posteriori 
es, en general, difícil de calcular y por ello la inferencia Bayasiana perdió
simpatía frente a los métodos frecuentistas. Sin embargo, el desarrollo de poder de cómputo
accesible junto con la teoría de los métodos de Monte Carlo que surgió a mediados del 
siglo revitalizaron la estadística Bayesiana. 

\subsubsection{Técnicas de Monte Carlo}

Dada una variable aleatoria con función de distribución $f$ condicionada, el valor 
esperado de una función de tal variable está dado por

\begin{equation}
    \mathbb E\left[h(x)|y\right]=\int h(x)f(x|y)dx
\end{equation} 

Esta integral sólo puede ser evaluada analíticamente en contados casos, por lo que es 
necesario utilizar métodos de aproximación, siendo los métodos de Monte Carlo uno de 
los más usados dado su facilidad de implementación e invarianza respecto a la 
dimensionalidad de la integral. 

\subsubsection{Algoritmo de Metropolis-Hastings}

Dada una densidad objetivo $f$, el algoritmo de Metropolis-Hastings contruye una 
cadena de Markov con distribución límite $f$, es decir que, a la larga, al tomar 
muestras de esta cadena se está muestreando de una aproximación a la verdadera
distribución objetivo. Para implementar esta aproximación se necesita de una 
función de transición $q:S\times S\to \mathbb R$. La construcción de esta 
cadena se presenta en el algoritmo \ref{alg:Metropolis-Hastings}. La validez de este 
algoritmo ha sido ya demostrada, véase \cite{rinconElementosSimulacionEstocastica2024}. 

\begin{algorithm}
\caption{Muestreo de Metropolis-Hastings}
\label{alg:Metropolis-Hastings}
\begin{algorithmic}[1]
\vspace{0.2cm}
\Require \parbox[t]{13cm}{
    Función de densidad objetivo $f$, función de transición $q(x,y)$,
    estado inicial $x_0\in S$
}
\vspace{0.2cm}
\Ensure 
\parbox[t]{13cm}{
    Muestra de una cadena de Markov con distribución estacionaria $f$
}
\vspace{0.2cm}
\State $x \leftarrow x_0$
\For{$t\in 1,2,...$}
    \State $y \sim q(x, y)$
    \State $\alpha = \min\left\{\frac{f(y)}{f(x)}\cdot \frac{q(y,x)}{q(x,y)},\ 1\right\}$
    \State $u \sim U(0,1)$
    \If{$u \leq \alpha$}
        \State $x \leftarrow y$
    \EndIf
\EndFor
\end{algorithmic}
\end{algorithm}





\subsection{Procesos de Markov parcialmente observados}

Se ha construido esta parte del trabajo a partir de \cite{kingStatisticalInferencePartially2016}.

Sea $\theta\in\Theta\subseteq \mathbb R^K$. Para cada valor fijo de $\theta$, el proceso estocástico 
$X = \{X(t,\theta)\}_{t\in T}$ representa un sistema dinámico que no puede ser observado 
directamente y cuya evolución futura depende únicamente del valor anterior inmediato. Dado un vector de 
parámetros fijo y un tiempo $\tau_i$, $i=0,...,n$, se usarán las notaciones $X_i$ para $X(t_i, \theta)$
y $X_{i:j}$ para $(X_i, X_{i+1},...,X_{j})$, respectivamente. La función de densidad del sistema en el 
tiempo actual condicionado a la evolución pasada cumple que
\begin{equation}\label{eq:MarkovProperty}
    f_{X_{j}|X_{0},...,X_{j-1}}(x_{{m}}|\ x_{0},...,x_{j-1},\theta) = f_{X_{j}|X_{j-1}}(x_{j}|x_{j-1}, \theta) = 
    \varphi(x_j|x_{j-1},\theta),
\end{equation}
para $j=1,...,m$. Es decir, $X$ es un proceso de Markov.

El proceso $X$ sólo puede ser observado indirectamente a través de otro proceso $Y = \{Y(t, \theta)\}_{t\in T_{obs}}$.
Sea $t_0\in T$ el tiempo inicial del proceso X y sean $T_{obs}=\{t_i\in T,\ i=1,...,n\}$ los tiempos en los 
que se ha observado el proceso Las variables aleatorias, tales que $t_0\leq t_1<...<t_n.$ Las variables aleatorias 
observables $Y_{1:n}$ son condicionalmente independientes dadas $X_{0:n}$ y, además, el valor actual del proceso
$Y$ depende únicamente del estado actual del proceso $X$. Para $i=1,...,n$ se cumple que 
\begin{equation}\label{eq:ObservationProbability}
    f_{Y_i|Y_{1:i-1},X_{0:i}}(y_{i}|\ y_{1:i-1}, x_{0:i}, \theta) = f_{Y_i|X_i}(y_i| x_i, \theta) = \psi(y_i|x_i, \theta).
\end{equation}
Finalmente, al tiempo $t_0$, el estado inicial del sistema está sujeto a la distribución inicial
\begin{equation}\label{eq:InitialDistribution}
    f_{X_0}(x_{0}|\theta) = \mu(\theta).
\end{equation}

\begin{definition}
    Dado un vector de parémetros $\theta \in \Theta \subseteq \mathbb R^K$, un sistema dinámico latente 
    $X$ que cumple \ref{eq:MarkovProperty} y que tiene distribución inicial \ref{eq:InitialDistribution}
    al tiempo $t_0$, del cual, además, se han recolectado las observaciones $y^* = (y_1^*,...,y_n^*)$ en 
    los tiempos $t_0\leq t_1<...<t_n$ de un modelo \ref{eq:ObservationProbability}, es un \textit{proceso de Markov 
    parcialmente observado} y, generalmente, se denota de la siguiente manera
    \begin{align}\label{eq:POMP}
       \begin{split}
        X_{i}& \sim \varphi(\cdot | X_{i-1},\theta)\\
        Y_{i}& \sim \psi(\cdot | X_i,\theta)\\
        X_0& \sim \mu(\theta).
    \end{split} 
    \end{align}
    para $i = 1,...,n$.
\end{definition}

En la literatura, a \ref{eq:POMP} también se le conoce como \textit{state-space model} 
o \textit{hidden Markov model}. En este trabajo nos referiremos a este modelo como un POMP, por
sus siglas en inglés \textit{partially observed Markov Process}. Algunas de las aplicaciones de esta clase de modelos 
pueden ser consulatadas en \cite{sarkkaBayesianFilteringSmoothing2023}, estas 
abarcan los campos de la navegación, la ingeniería, telecomunicaciones, física y 
otros más. En la epidemiología, los procesos de Markov parcialmente observados han sido 
utilizados para introducir y manejar la incertidumbre en dinámicas de enfermedades infecciosas.

Dado que la mayoría de las ocaciones el vector de parámetros es desconocido, la tarea se ha de centrar entonces 
en inferir tal vector. Para ello, dadas las observaciones del proceso, se tiene que la función de log-verosimilitud 
es
\begin{equation}
    \ell(\theta) = \log f_{Y_{1:n}}(y^*, \theta) =  \log f_{Y_1}(y^*_1|\theta) + \sum_{i=2}^n \log f_{Y_i|Y_{1:i-1}}(y^*_i|y^*_{1:i-1},\theta).
\end{equation} 

Las características del modelo POMP permiten representar cada una de las probabilidades condicionales anteriores de la siguiente forma
\begin{equation}
    \begin{split}
        f_{Y_i|Y_{1:i-1}}(y^*_i|y^*_{1:i-1},\theta) &= \int f_{Y_i|X_i}(y^*_i|x_i,\theta)f_{X_i|Y_{1:i-1}}(x_i|y^*_{1:n-1},\theta)dx_i \\
        f_{Y_1}(y^*_1|\theta) &= \int f_{Y_1|X_1}(y^*_1|x_1,\theta)f_{X_1}(x_1|\theta)dx_1.
    \end{split}
\end{equation}

Como es de esperarse, la log-verosimilitud no tiene una forma analítica para un vasto número de modelos, sin embargo, puede
ser aproximada a través del método de Monte Carlo Secuencial (algoritmo \ref{alg:BF}), que construye una 

Los algoritmos de filtrado de partículas son aproximaciones de Monte Carlo al problema 
del filtrado. Fueron desarrollados para atacar problemas no lineales para los cuales 
otros algoritmos populares, como el filtrado de Kalman, no están diseñados. En particular, 
el algoritmo Bootsrap filter (algoritmo \ref{alg:BF}) es la versión básica de los algoritmos 
de fitlrado de partículas y es suficiente para las necesidades de este trabajo. 
Para conocer más algoritmos de esta clase, y su sustentación teórica, se refiere a 
\cite{sarkkaBayesianFilteringSmoothing2023}.

\begin{algorithm}
\caption{Bootstrap filter}
\label{alg:BF}
\begin{algorithmic}[1]
\vspace{0.2cm}
\Require \parbox[t]{13cm}{
    Distribución de transición $f$, distribución de medida $g$, distribución 
    inicial $\mu$, parámetros $\theta$, observaciones $Y_{1:T}$, número 
    de partículas $N$
}
\vspace{0.2cm}
\Ensure 
\parbox[t]{13cm}{
    Estimaciones del valor esperado de los estados $X_{1:T}$
}
\vspace{0.2cm}
\State $x_0^{(i)} \sim \mu(x,\theta)$, $i=1,...,N$
\For{$t\in 1,2,...,T$}
    \State $x_t^{(i)}\sim f(x_t|x_{t-1}^{(i)})$
    \State $w_{t}^{i} \propto g(y_t|x_t^{(i)})$ 
    \State $x_t^{(i)} \sim resampling(x_t,w_t)$
    \State $\mathbb E\left(h(x_t)\right) \approx \sum_{i=1}^N w_t^{i}\cdot h(x_t^{(i)})$
\EndFor
\end{algorithmic}
\end{algorithm}

Nótese que en el algoritmo $\ref{alg:BF}$, el símbolo $\propto$ implica dos tareas, a saber,
la asignación del valor y la posterior normalización. Además, el método de remuestreo varía
según distintas implementaciones. 


