\section{Modelo SEIR}

Según las propias características de cada enfermedad cuya dinámica 
pretende modelarse, el modelo SIR, estocástico o determinista, ha 
tenido que ser ajustado para reflejar tales características. 

(acá hay que listar ejemplos).

En el caso del COVID-19 (citar más ejemplos del covid). En particular, 
se ha adoptado el modelo SEIR estocástico de \textcite{baltazar-lariosMaximumLikelihoodEstimation2024},
que se obtiene al perturbar la tasa de muerte natural $\mu$, contrario a la metodología 
clásica de perturbar la tasa de infección del modelo. La ventaja de este enfoque es que obtienen 
una verosimilitud manejable que pueden maximizar para inferir los parámetros del modelo. La 
ecuación diferencial estocástica que rige su modelo es 
\begin{align}\label{eq:SEIRstochastic}
    \begin{split}
        f_\beta(t) &= \beta_s I^s_t + \beta_aI^a _t\\
        dS_t &= (\mu + \gamma R_t - (f_\beta(t) + \mu)S_t)dt + \sigma(1-S_t)dW_t\\
        dE_t &= (f_\beta(t)S_t - (\kappa + \mu)E_t)dt - \sigma E_tdW_t \\
        dI^a_t &= (p\kappa E_t - (\alpha_a+\mu)I^a_t)dt - \sigma I^a_tdW_t\\
        dI^s_t &= ((1-p)\kappa E_t - (\alpha_s + \mu)I_s)dt - \sigma I^s_t dW_t\\
        dR_t &= (\alpha_a I^a_t + \alpha_s I_t^s - (\gamma + \mu)R_t)dt - \sigma R_t dW_t
    \end{split}
\end{align}







