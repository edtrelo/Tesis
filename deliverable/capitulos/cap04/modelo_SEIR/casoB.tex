\section{Caso B: pandemia de COVID-19}

A principios del año 2020, el virus Sars-CoV-2 provocó una pandemia a nivel global.

\subsection{Datos disponibles}

En México, se tiene acceso a los datos diarios de casos confirmados, sospechosos, negativos 
y de defunciones, tanto a nivel municipal como estatal, a partir del 26 de febrero de 2020.
Si bien el sitio oficial presenta un tablero donde también se reportan los casos activos, no 
se tiene acceso a una serie histórica. Para estimar el número de casos activos por día ($I_t$), 
se ha seguido una de las metodologías que se describe en \cite{alvarezEstimatingActiveCases2021}: 
se toma como suposición que cada nuevo caso es activo durante un tiempo fijo, siendo este la 
suma de las medianas de los periodos de  recuperación post prueba (7 días) y el tiempo en que los 
pacientes se sintieron enfermos antes de salir positivo (3 días). Claro, esta estimación está sujeta 
al inherente error de medición de nuevos contagios. En 