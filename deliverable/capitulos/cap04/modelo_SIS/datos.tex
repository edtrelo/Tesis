\subsection{Datos}

El tipo de enfermedades modelables con el modelo SIS son, generalmente, enfermedades 
que se reportan de manera anual (probablemente sea buena idea citar reportes anuales 
de estas enfermedades de algunos países). Esto implica que la escala temporal es anual,
así que no se puede ignorar la dinámica propia de la población, en particular, en una escala
anual las poblaciones crecen o decrecen, pero difícilmente se mantienen igual. Los modelos 
epidemiológicos con demografía clásicos usan las tasas de natalidad y mortalidad para plasmar
de mejor manera los cambios en las poblaciones (hay que hablar de ejemplos). En tales modelos 
la consecuencia es que las poblaciones ya no son constantes, a diferencia de lo que sucede 
con el modelo \ref{eq:modeloSIS}. Para preservar a la población, el modelo supone que la tasa
de natalidad es igual a la tasa de mortalidad. La decisión de seguir usando el modelo 
\ref{eq:modeloSIS} en vez de desarrollar los mismos resultados teóricos que obtienen 
\textcite{grayStochasticDifferentialEquation2011} para un nuevo modelo que descarte la igualdad 
en las tasas de natalidad y mortalidad se debe a que el modelo resultante tendría una matriz 
de covarianza no positiva definida, y ello implicaría no poder hacer uso de ciertos métodos 
de estimación, como la verosimilitud simulada y la aproximación por esquema de simulación. Para 
usar el modelo de \cite{grayStochasticDifferentialEquation2011} para enfermedades longevas 
se deben seleccionar horizontes temporales de países en particular en los que la población no 
tuvo un cambio notorio. Ejemplos de esta situación son Italia en los años noventa y España en la 
década pasada.

Según los datos del Instituto Nacional de Estadística de España, 
la población en España aumentó tan solo un 1.47\% entre el primer trimestre de 2009 y el primer trimestre 
de 2019. En la población mayor a 14 años, este incremento fue del 1.59\%. Este dato sugiere que puede 
considerarse constante a la población española durante este horizonte temporal. Para fines del modelo, 
$N$ será definida como la media de las poblaciones en España mayores a 14 años reportadas entre 2009 y 2019.
El mismo instituto también reporta la evolución anual de la esperanza de vida al nacer en España. Se 
tomará a $\mu^{-1}$ como la media de estas esperanzas de vida en el periodo de estudio. Además,
el Centro Nacional de Epidemiología, a través de la Unidad de vigilancia de VIH, ITS y hepatitis B y C,
reportó en \cite{unidaddevigilanciadevihitsyhepatitisbycVigilanciaEpidemiologicaInfecciones2024} la incidencia
de infección gonocócica (gonorrea) por año en España, desde 1995 hasta 2023. 
Para el año 2023, la población menor a 15 años que presentó algún caso de infección fue un 0.2\% del total 
de los casos. Si se hace la suposición de que esto mismo sucede en todo el periodo, entonces puede seleccionarse 
a los habitantes de España mayores a 14 años como la población a estudiar. El modelo describe la dinámica de la 
prevalencia de la enfermedad, no de la incidencia. Para estimar el número de casos activos en cualquier 
momento del año, se toma una duración promedio de infecciosidad fija y se multiplica por la incidencia. 
Según tal referencia, una persona infectada es infecciosa por las dos semanas, en promedio, en que tardan 
en surgir los síntomas y la semana que el tratamiento tarda en hacer efecto. Los datos reales del INE y el CNE, junto 
con la estimación de prevalencia de la infección gonocócica pueden consultarse en la tabla \ref{tab:DatosEspana}. 
Los valores de los parámetros del modelo que se considerarán fijos están en la tabla \ref{tab:ParametrosFijosSIS}.

\begin{table}[!ht]
\centering
\csvreader[
    tabular={
        >{\raggedright\arraybackslash}m{2cm}
        >{\raggedright\arraybackslash}m{2.5cm}
        >{\raggedright\arraybackslash}m{2.5cm}
        >{\raggedright\arraybackslash}m{2.5cm}
        >{\raggedright\arraybackslash}m{2.5cm}
    },   
    table head=\toprule
        \textbf{Año} & \textbf{Población} & \textbf{Expectativa de vida} & \textbf{Incidencia} & \textbf{Prevalencia estimada} \\ \midrule,
    table foot = \bottomrule
]{../data/datos_limpios/modelo_SIS/espanadata.csv}{}%
{\csvcoli & \num[round-precision=0]{\csvcolii} & \csvcoliii & \num[round-precision=0]{\csvcoliv} & \num{\csvcolv}}
\caption{Distintos datos demográficos usados para establecer el modelo epidemiológico: (de izquierda a derecha)
    población mayor a 14 años, esperanza de vida al nacer, incidencia de infección gonocócica reportada y estimación de prevalencia, 
    por año, en España. Datos del Instituto Nacional de Estadística y el Centro Nacional de Epidemiología.}
\label{tab:DatosEspana}
\end{table}

Dada la baja disponibilidad de datos, se eligió usar una escala decenal, es decir, el tiempo $t=1$ representa una década. Esto 
responde también a la necesidad de contar con un tiempo de paso $\Delta t$ pequeño. Bajo el supuesto que $t=1$ es una década, 
entonces se tienen observaciones en los tiempos $t=0, 0.1, 0.2,...,1.0$, implicando que $\Delta t = 0.1$. 

\begin{table}[!ht]
    \centering
    \begin{threeparttable}
        \begin{tabular}{
            >{\raggedright\arraybackslash}m{2.5cm}
            >{\raggedright\arraybackslash}m{3.5cm}
            >{\raggedright\arraybackslash}m{3.5cm}
            >{\raggedright\arraybackslash}m{2.5cm}
        }
        \hline
        \textbf{Parámetro} & \textbf{Descripción} & \textbf{Valor} & \textbf{Fuente} \\ \hline
        $N$ & Tamaño de la población & 39,591,836 & INE\tnote{†} \\ 
        $\mu^{-1}$ & Expectativa de vida & (8.267-1.5) décadas & INE\tnote{†} \\
        $\gamma^{-1}$ & Tiempo promedio de recuperación & $\frac{21}{365.25 \times 10}$ décadas &  \\ \hline
        \end{tabular}
        \begin{tablenotes}
            \footnotesize
            \item[†] Instituto Nacional de Estadística de España.
        \end{tablenotes}
        \caption{Parámetros fijos del modelo.}
        \label{tab:ParametrosFijosSIS}
    \end{threeparttable}
\end{table}

Dados los datos mostrados en la tabla \ref{tab:DatosEspana}, se define a la prevalencia estimada normalizada en el año 2009 + $10t$ años 
como las las observaciones del proceso $\hat{i}_t$, para $t=0.0,0.1,...,1.0$.

\subsection{Valores iniciales de los parámetros a estimar}

Habiendo supuesto que la tasa de mortalidad y recuperación son conocidas, entonces los parámetros a estimar a partir de los datos 
son el valor esperado de la tasa de infección $\beta$ y la intensidad  del ruido $\sigma$. Los algoritmos que se usarán para 
realizar la inferencia de estos parámetros requieren de valores iniciales para iniciar los procesos de optimización. 

Para la tasa de transmisión, considérese la parte determinista del modelo SIS normalizado, donde, dando un paso discreto,
la diferencia entre la prevalencia al tiempo $t$ y $t+\Delta$ está dada por
\begin{equation}
    i_{t+\Delta} - i_t = (\beta N i_t (1 - I_t) - (\mu + \gamma) i_t)\Delta.
\end{equation}
Las estimaciones de prevalencia están separadas por un año de diferencia. Al trabajar en una escala decenal, 
se toma a $\Delta = 0.1$. Luego, definiendo a $\hat{y}_t=(\hat{i}_{t+1} - \hat{i}_t) + (\mu + \gamma)\hat{i}_t$ y $\hat{x}_t=
N\hat{i}_t(1-\hat{I}_t)$, para $t = 0.0,...,0.9$, se ajusta un modelo de regresión lineal sin intercepto $\hat{y}\sim \hat{x}+0$ y se
define a $\beta_0$, el valor inicial de $\beta$, como la pendiente estimada de tal modelo. 

De la definición \ref{eq:QV}, la varianza cuadrática empírica de $I$ en el intervalo $[0, 10]$ es aproximadamente igual a 
\begin{equation}
    [i]^{(n)}_{10} = \sum_{t=0}^{9} (\hat{i}_{(t+1)/10} - \hat{i}_{t/10})^2 \approx \sigma^2  N^2 Q\left(\mathcal L\left(\left\{(t, \hat{i}_t^2(1-\hat{i}_t)^2): 
    t=0.0,...,1.0\right\}\right), 0.0, 1.0\right),
\end{equation}
donde $\mathcal L(\{(x_1,y_1),...,(x_n,y_n)\})$ es una interpolación lineal de $\{(x_1,y_1),...,(x_n,y_n)\}$ y $Q(f,a,b)$ es una
aproximación numérica de la integral definida de $f$ en $[a,b]$. Despejando $\sigma$, se define a $\sigma_0$, el valor inicial de 
$\sigma$, como
\begin{equation}
    \sigma_0 = \left(\frac{\sum\limits_{t=0}^{9} (\hat{i}_{(t+1)/10} - \hat{i}_{t/10})^2}{N^2 Q\left(\mathcal L\left(\left\{(t, \hat{i}_t^2(1-\hat{i}_t)^2): 
    t=0.0,...,1.0\right\}\right), 0.0, 1.0\right)}\right)^{\frac{1}{2}}.        
\end{equation}

Para los datos disponibles, los valores iniciales resultantes son 
$\beta_0 = 4.403 \times 10^{-6}$ y  $\sigma_0 = 1.653283 \times 10^{-8}$. Véase \texttt{modeloSIS/initial\_conditions}
para los detalles de la implementación.