\section{Modelo SIR}

El modelo SIR fue introducido por tal. Se trata de un 
sistema de ecuaciones diferenciales ...\\


Meter ruido...\\ 


La demostración de la existencia y unicidad de la solución
de este sistema de ecuaciones diferenciales estocásticas 
puede consultarse en \cite{aliStochasticModelingInfluenza2024}. 
Si consideramos una población cerrada, es decir que $\mu = 0$,
entonces $dS_t + dI_t + dR_t = 0$, por lo que $S_t+I_t+R_t = C$, con 
$C$ una constante. Si los valores iniciales son tales que su suma 
es igual a uno, entonces aseguramos que $S_t+I_t+R_t=1$, para todo 
$t\geq 0$.
