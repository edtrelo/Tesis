\section{Modelo SIR}

El modelo SIR fue introducido por tal. Se trata de un 
sistema de ecuaciones diferenciales ...\\


Meter ruido...\\ 


La demostración de la existencia y unicidad de la solución
de este sistema de ecuaciones diferenciales estocásticas 
puede consultarse en \cite{aliStochasticModelingInfluenza2024}. 
Si consideramos una población cerrada, es decir que $\mu = 0$,
entonces $dS_t + dI_t + dR_t = 0$, por lo que $S_t+I_t+R_t = C$, con 
$C$ una constante. Si los valores iniciales son tales que su suma 
es igual a uno, entonces aseguramos que $S_t+I_t+R_t=1$, para todo 
$t\geq 0$.

\subsection{Modelo matemático}

Sea $S_t, I_t, R_t$ el número de personas dentro de los compartimientos 
susceptibles, infectados y recuperados, respetivamente, al tiempo $t\geq 0$. 
Si $S_0+I_0+R_0 = N$, entonces la evolución temporal de los procesos 
$S_t, I_t$ y $R_t$ está sujeta a la siguiente ecuación
\begin{align}
    dS_t &= -\beta S_tI_t dt - \sigma_\beta S_tI_tdW^{(1)}_t, \\
    dI_t &= (\beta S_tI_t - \gamma I_t)dt + \sigma_\beta S_tI_tdW
    _t - \sigma_\gamma I_tdW^{(2)}_t,\\
    R_t &= N - (S_t-I_t).
\end{align}
Las ecuaciones diferenciales estocásticas de este modelo pueden ser 
expresadas en forma matricial del estilo $dX_t = f(X_t,\theta)dt + g(X_t,\theta)dW_t$, como 
\begin{equation}\label{eq:SIRmatricial}
    dX_t = 
    \begin{bmatrix}
        -\beta S_t I_t\\
        \beta S_tI_t - \gamma I_t
    \end{bmatrix}dt + \begin{bmatrix}
        -\sigma_\beta S_tI_t & 0\\
        \sigma_\beta S_tI_t & -\sigma_\gamma S_tI_t
    \end{bmatrix}dW_t,
\end{equation}
donde $X_t = (S_t, I_t)'$ y $\theta = (\beta, \gamma, \sigma_\beta, \sigma_\gamma)'$.

\begin{proposition}
    Sean $S_t,I_t,\beta,\gamma,\sigma_\beta, \sigma_\gamma\geq 0$. Entonces la 
    matriz de difusión de la ecuación \ref{eq:SIRmatricial} no es definida positiva.
\end{proposition}

\subsection{Reconstrucción de variables observadas}

En el contexto del estudio de las enfermedades infecciosas, las observaciones de la 
evolución de la transmisión son escasas y, la mayoría de las veces, indirectas, pero 
eso no ha sido impedimento para tratar de entender esta dinámica a partir de cualquier
dato que se haya logrado recolectar.

Supóngase que se tiene acceso a un conjunto de observaciones de nuevas infecciones 
diarias de manera consecutiva, entre los días $t_0$ y $t_n$. Se denotará el 
número de nuevos casos de infección el día $t_i$ como $J^{o}_{t_i}$, para
$i=0,...,n$. Si la tasa de recuperación $\gamma$ es conocida (o se tiene un buen estimado 
de su valor real), entonces se define a $\gamma' = \lfloor \frac{1}{\gamma}\rfloor$. 
Luego, se estima el número de infectados y recuperados al tiempo $t$ como
\begin{equation}
    (\hat{I}_{t}, \hat{R}_t) = \begin{cases}
        \left(\sum_{\tau = t_0}^t J_\tau^o, 0\right), &\text{si } t< \gamma',\\
        \sum\limits_{\tau = t-\gamma'}^tJ_\tau^o, &\text{si }t\geq \gamma',
    \end{cases}
\end{equation}
donde $t=t_0,...,t_n$. Además, si se ha establecido el supuesto que la enfermedad es tal que
no hay reinfección entonces se puede estimar al número de susceptibles como
\begin{align}
    \hat{S}_t^o = N - \sum_{\tau = t_0}^t J_t^o,\\
    \hat{R}_t^o = N - (\hat{S}_t^o + \hat{I}_t^o),
\end{align}
donde $N$ es el número de habitantes.

Las estimaciones anteriores suponen que las observaciones de los nuevos casos de infectados 
son directas, sin embargo, estas observaciones generalmente solo representan una porción 
$\rho \in (0,1]$ de los casos reales.


\subsection{Valores iniciales para estimación}

En esta sección se construirán los estimadores iniciales para los métodos de estimación que 
así lo requieran. Dado que algunos métodos de optimización son sensibles a los valores 
iniciales, es importante tener acceso a valores iniciales plausibles.

Sea $\mathcal T=\{t_0,...,t_n: 0\leq t_0<...<t_n\}$ un  índice de tiempos discretos y 
$X^{obs} = \{X^{obs}_{t} = \varphi(S_{t}^{obs}, I_{t}^{obs}): t \in \mathcal T\}$ un conjunto 
de observaciones de alguna función $\varphi: \mathbb R^2 \to \mathbb R^\kappa$ del proceso cuyos
parámetros quieren ser estimados. Por ejemplo, si las observaciones son completas y directas, 
entonces $\varphi(x,y)=(x,y)$; pero si solo se tiene acceso al número de infectados reportados 
en cierto instante de tiempo entonces $\varphi(x,y)=\rho y$, donde $\rho\in (0, 1]$ es una tasa 
de identificación. 

\subsubsection{Coeficientes de difusión}

En la sección \ref{sec:cap02_teoria} se introdujo la definición de variación cuadrática. (insertar 
interpretación de la variación cuadrática). Usando el modelo \ref{eq:SIRmatricial}, se obtiene 
que los parámetros $\sigma_\beta$ y $\sigma_\gamma$ cumplen con la siguiente relación:

\begin{align}\label{eq:QVSIR}
    \begin{split}
        \lim\limits_{n\to\infty}\sum_{k=1}^{2^n}(S_{t^{(n)}_{k}}-S_{t^{(n)}_{k-1}})^2& = 
        \sigma_\beta^2\int_s^t (S_{\tau}I_\tau)^2d\tau,\\
        \lim\limits_{n\to\infty}\sum_{k=1}^{2^n}(I_{t^{(n)}_{k}}-I_{t^{(n)}_{k-1}})^2& = 
        \sigma_\beta^2\int_s^t (S_{\tau}I_\tau)^2d\tau + \sigma_\gamma^2\int_s^tI_\tau^2d\tau.
    \end{split}
\end{align}

Para construir los estimadores, considérese una observación discreta del proceso 
$X^{obs} = \{X^{obs}_{t_1},...,X^{obs}_{t_n}\}$, con $X^{obs}_{t_j}=(S_{t_j}^{obs}, I_{t_j}^{obs})'$. 
Entonces, al tomar una discretización de la integral y la suma, se obtiene la siguiente aproximación 
a la ecuación \ref{eq:QVSIR}:

\begin{align}
    \begin{split}
    (S^{obs}_{t_k}-S^{obs}_{t_{k-1}})^2 \approx \sigma_\beta^2 Q_{t_0, t_n}\left((L_{S^{obs}}(t)L_{I^{obs}}(t))^2\right),\\
    (I^{obs}_{t_k}-I^{obs}_{t_{k-1}})^2 \approx \sigma_\beta^2 Q_{t_0, t_n}\left((L_{S^{obs}}(t)L_{I^{obs}}(t))^2\right)+
    \sigma_\gamma^2 Q_{t_0, t_n}\left((L_{I^{obs}}(t))^2\right),
    \end{split}
\end{align}

donde $Q_{s, t}(f)$ es la cuadratura de la función $f$ en el intervalo $(s,t)$ y $L_{A}$ es la interpolación lineal 
del conjunto $A$, que consiste de tiempos y las observaciones correspondientes en cada tiempo.

\subsubsection{Tasa de infección}

Considérese la versión determinista y discreta del modelo \ref{eq:SIRmatricial}. La ecuación en 
diferencias para los infectados es

\begin{equation}
    I_{t_k}-I_{t_{k-1}}=\beta S_{t_{k-1}}I_{t_{k-1}} - \gamma I_{t_{k-1}}.
\end{equation}

Si el valor de $\gamma$ se conoce de antemano, podemos estimar a $\beta$ en el intervalo de tiempo 
$(t_{k-1},t_{k})$ como

\begin{equation}
    \hat{\beta}_k = \frac{ I_{t_k}-I_{t_{k-1}} - \gamma I_{t_{k-1}}}{S_{t_{k-1}}I_{t_{k-1}}}.
\end{equation}


\subsubsection{tasa de recuperacion}