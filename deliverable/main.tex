\documentclass[letterpaper,12pt,spanish]{book} 
\usepackage[
    es-tabla,es-nodecimaldot
    ]{babel} %es-nodecimaldot es para usar punto como separador
\usepackage{lipsum} % Lorem Ipsum dummy text
\usepackage{amssymb, amsmath, amsthm, csquotes}
\usepackage[
    top=1.5in, left=1.25in, right=1.25in, bottom=1.5in
    ]{geometry} % Tamaño de papel y margen
\setlength{\parskip}{10pt} % Espacio entre párrafos
\usepackage{import}
\usepackage{algorithm}
\usepackage{algpseudocode}
\floatname{algorithm}{Algoritmo}
\renewcommand{\algorithmicrequire}{\textbf{Input:}}
\renewcommand{\algorithmicensure}{\textbf{Output:}}
\usepackage{etoolbox}
\makeatletter
\patchcmd{\ALG@numberline}{\hfil\relax}{}{}{}
\renewcommand{\alglinenumber}[1]{\footnotesize #1\hspace{1em}} % sin punto, con espaciado
\makeatother
%------------------------------------------
% PORTADA
%------------------------------------------
\import{cover}{cover.tex}
%------------------------------------------
% ESTILO DE LAS PÁGINAS
%------------------------------------------
\usepackage{fancyhdr}
\pagestyle{fancy}
\fancyhf{} % Limpiar encabezado y pie
% Encabezados personalizados
% Página par (E: even), encabezado derecho (R: right)
% Página impar (O: odd), encabezado izquierdo (L: left)
\fancyhead[RE]{
    \textbf{\nouppercase{Capítulo \thechapter. \leftmark}}
    } 
\fancyhead[LO]{
    \textbf{\nouppercase{\thesection\ \rightmark}}
    }          
% Pie de página centrado
\fancyfoot[C]{\thepage}
% Líneas
\renewcommand{\headrulewidth}{0pt}
\renewcommand{\footrulewidth}{0pt}
% Actualizar las marcas
\renewcommand{\chaptermark}[1]{\markboth{#1}{}}
\renewcommand{\sectionmark}[1]{\markright{#1}}
\setlength{\headheight}{14.5pt}
\usepackage{emptypage} % Evita numerar la página que deja 
% en blanco el salto de capítulo
%------------------------------------------
% MATHEMATICAL ENVIROMENTS
%------------------------------------------
\newtheorem{theorem}{Teorema}[section]
\newtheorem{definition}{Definición}[section]
\newtheorem{proposition}{Proposición}[section]
\newtheorem*{example}{Ejemplo}
\newcommand*\btheta{\boldsymbol{\theta}}
%------------------------------------------
% PREÁMBULO PARA REFERENCIAS
%------------------------------------------
\usepackage[backend=biber,style=ieee]{biblatex}
\addbibresource{references.bib} 
\usepackage[hidelinks]{hyperref} % Necesario para usar la solución de capítulos sin número: https://tex.stackexchange.com/questions/11668/adding-unnumbered-sections-to-toc
%\renewcommand{\contentsname}{}
%------------------------------------------
% INICIO DE DOCUMENTO 
%------------------------------------------
\begin{document} % Inicia el documento

\maketitle % Crea portada
\tableofcontents % Index 
\markboth{}{}


%\chapter{Motivación}
%\subfile{capitulos/motivacion}

\chapter*{Introducción}\label{cap:intro}
\addcontentsline{toc}{chapter}{\nameref{cap:intro}}
%\subfile{capitulos/introduccion}

\chapter
{Principios de probabilidad e inferencia estadística}
\label{cap:01}
\section{Teoria}\label{sec:cap02_teoria}

\subsection{Variación cuadrática}

La siguiente definición ha sido tomada de \cite{fuchsInferenceDiffusionProcesses2013}.

\begin{definition}
    Sea $X$ un proceso de Itô observado continuamente en el intervalo $[s,t]$, entonces
    la \textbf{variación cuadrática} del proceso en $[s,t]$ se define como
    \begin{equation}\label{eq:QV}
        [X,X]_{s,t} = \lim\limits_{n\to 0}\sum_{k=1}^{2^n}\left(X_{t_{k}^{(n)}}-X_{t_{k-1}^{(n)}}\right)
        \left(X_{t_{k}^{(n)}}-X_{t_{k-1}^{(n)}}\right)'=\int_s^t g(X_\tau,\theta)g(X_\tau,\theta)'d\tau,
    \end{equation}
    donde $t_k^{(n)}=s+k2^{-n}(t-s)$ for $k=0,...,2^n$, y la segunda igualdad es en probabilidad y 
    casi seguramente.
\end{definition}


\section{Solución numérica de EDEs}\label{sec:metodos_numericos}
\textcite{rackauckasDifferentialEquationsjlPerformantFeatureRich2017a}
Dado que la solución analítica de una ecuación diferencial estocástica 
es, en general, imposible de calcular, se tiene que recurrir a 
métodos  numéricos que aproximen esa solución. Hay una amplia 
variedad de métodos numéricos desarrollados para este fin, sin 
embargo se optó por usar un método que es fácil de implementar
si el proceso de difusión es multivariado, el método de Euler-Maruyama.
Otros métodos existen y, en general, tienen un mejor desempeño 
pero su implementación implica un mayor esfuerzo. Entre estos 
se encuentran los métodos de Milstein, Runge-Kutta.

\subsection{Convergencia y consistencia}

\begin{definition}
    Una aproximación a tiempo discreto $Y^\delta$, donde $\delta$ es el 
    tamaño de paso máximo, tiene un nivel de convergencia fuerte $\gamma$
    al proceso $X$ si existe una constante positiva $K$, que no depende de 
    $\delta$, y un $\delta_0>0$ tal que 
    $$\mathbb E\left(|X_t - Y_t^\delta|\right)\leq K\delta^\gamma$$
    para todo $t\geq 0$ y toda $\delta \in (0,\delta_0)$.
\end{definition}

\subsection{Método de Euler-Maruyama}
Si quiere obtenerse 
una aproximación de la solución a la ecuación tal 
en el intervalo $[t_0,T]$ tal que $X_0 = x_0$. Considérese la partición 
$t_0=\tau_0<\tau_1<...<\tau_n=T$ y defínase $\Delta_i=\tau_{i+1}-\tau_i$
para $i=0,...,n-1$. El proceso estocástico continuo $Y=\{Y_t\}_{t\geq t_0}^T$ 
se define, generalmente, como la interpolación lineal del conjunto de puntos 
$\{(\tau_i, Y_{\tau_i}):i=0,...,n)\}$ donde 
$Y_{\tau_0}=x_0$ y el resto satisfacen la relación iterativa
\begin{equation}\label{eq:EM}
Y_{\tau_{i+1}}=Y_{\tau_{i}}+f(X_{\tau_i},\tau_i)\Delta_i 
+ g(Y_{\tau_i},\tau_i)\Delta_i(W_{\tau_{i+1}}-W_{\tau_i})   
\end{equation}
para  $i=0,...,n-1$. En general, se toma una partición con puntos
equidistantes, es decir, $\Delta_i = \Delta = \frac{T-t_0}{n}$ y $\tau_i = \tau_0+i\Delta$. 
Para  simular este proceso, se usa la normalidad de los incrementos 
del movimiento Browniano (ver algoritmo \ref{alg:EM}). La paquetería 
\texttt{DifferentialEquations.jl} permite resolver una ecuación estocástica
con este esquema especificnado \texttt{EM()} como método y
un valor para $\Delta$ por medio del parámetro \texttt{dt}, siendo 
$\Delta=\frac{T-t_0}{n}$, para algún $n\in \mathbb N_+$, de modo que 
el valor en la última posición del vector arrojado por el algoritmo 
corresponda a $Y_{T}$.

\begin{algorithm}
\caption{Esquema de Euler-Maruyama}
\label{alg:EM}
\begin{algorithmic}[1]
\vspace{0.2cm}
\Require \parbox[t]{13cm}{
    Función de deriva $f$, función de difusión $g$, tamaño de paso 
    $\Delta$, lapso de tiempo $(t_0,T)$, valor inicial $x_0$
}
\vspace{0.2cm}
\Ensure 
\parbox[t]{13cm}{
    Aproximación de la solución a $dX_t=f(X_t,t)dt+g(X_t,t)W_t$ sujeta a 
    $X_{t_0}=x_0$ evaluada en 
    $t\in\{t_i=t_0 + i\Delta:\ t_i\leq T,\ i\in \mathbb N_+\}$
}
\vspace{0.2cm}
\State $d \leftarrow \dim(x_0)$
\State $n\leftarrow \lfloor(T-t_0)/\Delta\rfloor$
\State $Y \leftarrow [n+1,d]$
\State $Y[1,:]\leftarrow x_0$
\For{$i\in 1,...,n$}
    \State $\varepsilon \sim N_d(0, I_d)$
    \State $t\leftarrow t_0+i\Delta$
    \State $Y[i+1,:]\leftarrow f(Y[i,:],t)\cdot \Delta+g(Y[i,:],t)\cdot \Delta\cdot  \varepsilon$
\EndFor
\end{algorithmic}
\end{algorithm}

El método de Euler-Maruyama tiene un orden de convergencia de 
$\gamma = \frac{1}{2}$, mientras que los métodos de Milstein y 
Runge-Kutta tienen tal y tal, respectivamente 
\cite{fuchsInferenceDiffusionProcesses2013}.

\subsection{Esquema de Milstein}

De forma análoga que en el método de Euler-Maruyama, para construir la aproximación
de Milstein debe considerarse una partición $t_0=\tau_0<\tau_1<...<\tau_n=T$
equidistante, de forma que $\Delta_i=\tau_{i+1}-\tau_{i}$ 
sea una constante $\Delta$ para todo $i=0,...,n-1$. 
Entonces la componente $k$-ésima del proceso aproximación cumple la 
relación iterativa siguiente 
\begin{equation}\label{eq:Milstein}
    \begin{split}
        Y_{\tau_{i+1}}^{(k)}=&Y_{\tau_i}^{(k)}+f_k(Y_{\tau_{i}},\tau_i)\Delta
+ \sum_{j=1}^m g_{kj}(Y_{\tau_i},\tau_i)\Delta W_i^{(j)}\\
&+\sum_{j,l=1}^m\sum_{r=1}^dg_{rj}(Y_{\tau_i},\tau_i)
\left(\frac{\partial g_{kl}}{\delta x^{(r)}}(Y_{\tau_i},\tau_i) 
\right)\int\limits^{\tau_{i+1}}_{\tau_i} \int\limits^s_{\tau_i}dW_u^{(j)}dW_s^{(l)}
    \end{split}
\end{equation}
para $i=0,...,n-1$ y donde $\Delta W_{i}=W_{\tau_{i+1}}-W_{\tau_i}$. Para una 
dimensión, la implementación del método no tiene mayor complicación, 
sin embargo, en el caso multivariado el cálculo de las derivadas parciales 
de $g$ y de la doble integral representa un problema que puede atacarse 
desde distintos enfoques y que, en ciertas condiciones, se simplifica 
\cite{kloedenNumericalSolutionStochastic1992}. 
%Por ejemplo, cuando el coeficiente de difusión es diagonal 


\subsection{Esquema de Runge-Kutta-Milstein}
% pp. 373 kloeden 
Tal como en el caso determinista, el esquema de Milstein (eq. \ref{eq:Milstein})
puede ser modificado con el fin de evitar la evaluación de las derivadas del coeficiente
de difusión. Usando diferencias finitas se llega al esquema de Runge-Kutta-Milstein, 
referenciado en \cite{kloedenNumericalSolutionStochastic1992} como esquema fuerte de orden
1.0 explícito. Considerando de nuevo una partición equidistante, la $k$-ésima 
componente de esta aproximación cumple la siguiente relación iterativa
% Explicit Order 1.0 Strong Schemes
\begin{equation}\label{eq:Runge-Kutta-Milstein}
    \begin{split}
    Y_{\tau_{i+1}}^{(k)}=&Y_{\tau_i}^{(k)} + f_k(Y_{\tau_i},\tau_i)\Delta + 
    \sum_{j=1}^m g_{kj}(Y_{\tau_i},\tau_i)\Delta W_i^{(j)}\\ &
    + \frac{1}{\sqrt{\Delta}}\sum_{j,l=1}^m \left(g_{k,l}(\tilde Y_{j,\tau_i}, \tau_i)- g_{k,l}(Y_{\tau_i},\tau_i) 
    \right)\int\limits^{\tau_{i+1}}_{\tau_i} \int\limits^s_{\tau_i}dW_u^{(j)}dW_s^{(l)}\\
    \tilde Y_{j,\tau_i}=& Y_{\tau_i}+f(Y_{\tau_i},\tau_i)\Delta + g_{:,j}(Y_{\tau_i},\tau_i)\sqrt{\Delta}
    \end{split}
\end{equation}
% sigma_{:,j} j-ésima columna de sigma
para $i=0,...,n-1$ y $j=1,...,m$ y donde $\Delta W_{i}=W_{\tau_{i+1}}-W_{\tau_i}$ y $g_{:,j}$ es la $j$-ésima
columna de $g$. Esta aproximación es fuertemente consistente de orden $\gamma = 1.0$, si los
coeficientes son dos veces diferenciables y tales derivadas son uniformemente acotadas \cite{fuchsInferenceDiffusionProcesses2013}.
Este esquema está implementado en \texttt{DifferentialEquations.jl} como \texttt{RKMil} y requiere de
un tamaño de paso fijo.


\chapter{Ecuaciones diferenciales estocásticas y procesos de difusión}
\label{cap:02}
\section{Teoria}\label{sec:cap02_teoria}

\subsection{Variación cuadrática}

La siguiente definición ha sido tomada de \cite{fuchsInferenceDiffusionProcesses2013}.

\begin{definition}
    Sea $X$ un proceso de Itô observado continuamente en el intervalo $[s,t]$, entonces
    la \textbf{variación cuadrática} del proceso en $[s,t]$ se define como
    \begin{equation}\label{eq:QV}
        [X,X]_{s,t} = \lim\limits_{n\to 0}\sum_{k=1}^{2^n}\left(X_{t_{k}^{(n)}}-X_{t_{k-1}^{(n)}}\right)
        \left(X_{t_{k}^{(n)}}-X_{t_{k-1}^{(n)}}\right)'=\int_s^t g(X_\tau,\theta)g(X_\tau,\theta)'d\tau,
    \end{equation}
    donde $t_k^{(n)}=s+k2^{-n}(t-s)$ for $k=0,...,2^n$, y la segunda igualdad es en probabilidad y 
    casi seguramente.
\end{definition}


\section{Solución numérica de EDEs}\label{sec:metodos_numericos}
\textcite{rackauckasDifferentialEquationsjlPerformantFeatureRich2017a}
Dado que la solución analítica de una ecuación diferencial estocástica 
es, en general, imposible de calcular, se tiene que recurrir a 
métodos  numéricos que aproximen esa solución. Hay una amplia 
variedad de métodos numéricos desarrollados para este fin, sin 
embargo se optó por usar un método que es fácil de implementar
si el proceso de difusión es multivariado, el método de Euler-Maruyama.
Otros métodos existen y, en general, tienen un mejor desempeño 
pero su implementación implica un mayor esfuerzo. Entre estos 
se encuentran los métodos de Milstein, Runge-Kutta.

\subsection{Convergencia y consistencia}

\begin{definition}
    Una aproximación a tiempo discreto $Y^\delta$, donde $\delta$ es el 
    tamaño de paso máximo, tiene un nivel de convergencia fuerte $\gamma$
    al proceso $X$ si existe una constante positiva $K$, que no depende de 
    $\delta$, y un $\delta_0>0$ tal que 
    $$\mathbb E\left(|X_t - Y_t^\delta|\right)\leq K\delta^\gamma$$
    para todo $t\geq 0$ y toda $\delta \in (0,\delta_0)$.
\end{definition}

\subsection{Método de Euler-Maruyama}
Si quiere obtenerse 
una aproximación de la solución a la ecuación tal 
en el intervalo $[t_0,T]$ tal que $X_0 = x_0$. Considérese la partición 
$t_0=\tau_0<\tau_1<...<\tau_n=T$ y defínase $\Delta_i=\tau_{i+1}-\tau_i$
para $i=0,...,n-1$. El proceso estocástico continuo $Y=\{Y_t\}_{t\geq t_0}^T$ 
se define, generalmente, como la interpolación lineal del conjunto de puntos 
$\{(\tau_i, Y_{\tau_i}):i=0,...,n)\}$ donde 
$Y_{\tau_0}=x_0$ y el resto satisfacen la relación iterativa
\begin{equation}\label{eq:EM}
Y_{\tau_{i+1}}=Y_{\tau_{i}}+f(X_{\tau_i},\tau_i)\Delta_i 
+ g(Y_{\tau_i},\tau_i)\Delta_i(W_{\tau_{i+1}}-W_{\tau_i})   
\end{equation}
para  $i=0,...,n-1$. En general, se toma una partición con puntos
equidistantes, es decir, $\Delta_i = \Delta = \frac{T-t_0}{n}$ y $\tau_i = \tau_0+i\Delta$. 
Para  simular este proceso, se usa la normalidad de los incrementos 
del movimiento Browniano (ver algoritmo \ref{alg:EM}). La paquetería 
\texttt{DifferentialEquations.jl} permite resolver una ecuación estocástica
con este esquema especificnado \texttt{EM()} como método y
un valor para $\Delta$ por medio del parámetro \texttt{dt}, siendo 
$\Delta=\frac{T-t_0}{n}$, para algún $n\in \mathbb N_+$, de modo que 
el valor en la última posición del vector arrojado por el algoritmo 
corresponda a $Y_{T}$.

\begin{algorithm}
\caption{Esquema de Euler-Maruyama}
\label{alg:EM}
\begin{algorithmic}[1]
\vspace{0.2cm}
\Require \parbox[t]{13cm}{
    Función de deriva $f$, función de difusión $g$, tamaño de paso 
    $\Delta$, lapso de tiempo $(t_0,T)$, valor inicial $x_0$
}
\vspace{0.2cm}
\Ensure 
\parbox[t]{13cm}{
    Aproximación de la solución a $dX_t=f(X_t,t)dt+g(X_t,t)W_t$ sujeta a 
    $X_{t_0}=x_0$ evaluada en 
    $t\in\{t_i=t_0 + i\Delta:\ t_i\leq T,\ i\in \mathbb N_+\}$
}
\vspace{0.2cm}
\State $d \leftarrow \dim(x_0)$
\State $n\leftarrow \lfloor(T-t_0)/\Delta\rfloor$
\State $Y \leftarrow [n+1,d]$
\State $Y[1,:]\leftarrow x_0$
\For{$i\in 1,...,n$}
    \State $\varepsilon \sim N_d(0, I_d)$
    \State $t\leftarrow t_0+i\Delta$
    \State $Y[i+1,:]\leftarrow f(Y[i,:],t)\cdot \Delta+g(Y[i,:],t)\cdot \Delta\cdot  \varepsilon$
\EndFor
\end{algorithmic}
\end{algorithm}

El método de Euler-Maruyama tiene un orden de convergencia de 
$\gamma = \frac{1}{2}$, mientras que los métodos de Milstein y 
Runge-Kutta tienen tal y tal, respectivamente 
\cite{fuchsInferenceDiffusionProcesses2013}.

\subsection{Esquema de Milstein}

De forma análoga que en el método de Euler-Maruyama, para construir la aproximación
de Milstein debe considerarse una partición $t_0=\tau_0<\tau_1<...<\tau_n=T$
equidistante, de forma que $\Delta_i=\tau_{i+1}-\tau_{i}$ 
sea una constante $\Delta$ para todo $i=0,...,n-1$. 
Entonces la componente $k$-ésima del proceso aproximación cumple la 
relación iterativa siguiente 
\begin{equation}\label{eq:Milstein}
    \begin{split}
        Y_{\tau_{i+1}}^{(k)}=&Y_{\tau_i}^{(k)}+f_k(Y_{\tau_{i}},\tau_i)\Delta
+ \sum_{j=1}^m g_{kj}(Y_{\tau_i},\tau_i)\Delta W_i^{(j)}\\
&+\sum_{j,l=1}^m\sum_{r=1}^dg_{rj}(Y_{\tau_i},\tau_i)
\left(\frac{\partial g_{kl}}{\delta x^{(r)}}(Y_{\tau_i},\tau_i) 
\right)\int\limits^{\tau_{i+1}}_{\tau_i} \int\limits^s_{\tau_i}dW_u^{(j)}dW_s^{(l)}
    \end{split}
\end{equation}
para $i=0,...,n-1$ y donde $\Delta W_{i}=W_{\tau_{i+1}}-W_{\tau_i}$. Para una 
dimensión, la implementación del método no tiene mayor complicación, 
sin embargo, en el caso multivariado el cálculo de las derivadas parciales 
de $g$ y de la doble integral representa un problema que puede atacarse 
desde distintos enfoques y que, en ciertas condiciones, se simplifica 
\cite{kloedenNumericalSolutionStochastic1992}. 
%Por ejemplo, cuando el coeficiente de difusión es diagonal 


\subsection{Esquema de Runge-Kutta-Milstein}
% pp. 373 kloeden 
Tal como en el caso determinista, el esquema de Milstein (eq. \ref{eq:Milstein})
puede ser modificado con el fin de evitar la evaluación de las derivadas del coeficiente
de difusión. Usando diferencias finitas se llega al esquema de Runge-Kutta-Milstein, 
referenciado en \cite{kloedenNumericalSolutionStochastic1992} como esquema fuerte de orden
1.0 explícito. Considerando de nuevo una partición equidistante, la $k$-ésima 
componente de esta aproximación cumple la siguiente relación iterativa
% Explicit Order 1.0 Strong Schemes
\begin{equation}\label{eq:Runge-Kutta-Milstein}
    \begin{split}
    Y_{\tau_{i+1}}^{(k)}=&Y_{\tau_i}^{(k)} + f_k(Y_{\tau_i},\tau_i)\Delta + 
    \sum_{j=1}^m g_{kj}(Y_{\tau_i},\tau_i)\Delta W_i^{(j)}\\ &
    + \frac{1}{\sqrt{\Delta}}\sum_{j,l=1}^m \left(g_{k,l}(\tilde Y_{j,\tau_i}, \tau_i)- g_{k,l}(Y_{\tau_i},\tau_i) 
    \right)\int\limits^{\tau_{i+1}}_{\tau_i} \int\limits^s_{\tau_i}dW_u^{(j)}dW_s^{(l)}\\
    \tilde Y_{j,\tau_i}=& Y_{\tau_i}+f(Y_{\tau_i},\tau_i)\Delta + g_{:,j}(Y_{\tau_i},\tau_i)\sqrt{\Delta}
    \end{split}
\end{equation}
% sigma_{:,j} j-ésima columna de sigma
para $i=0,...,n-1$ y $j=1,...,m$ y donde $\Delta W_{i}=W_{\tau_{i+1}}-W_{\tau_i}$ y $g_{:,j}$ es la $j$-ésima
columna de $g$. Esta aproximación es fuertemente consistente de orden $\gamma = 1.0$, si los
coeficientes son dos veces diferenciables y tales derivadas son uniformemente acotadas \cite{fuchsInferenceDiffusionProcesses2013}.
Este esquema está implementado en \texttt{DifferentialEquations.jl} como \texttt{RKMil} y requiere de
un tamaño de paso fijo.

%\subfile{capitulos/resultados}

\chapter[Algunos métodos de inferencia para EDEs]
{Algunos métodos de inferencia para ecuaciones diferenciales estocásticas}
\label{cap:03}
\section{Teoria}\label{sec:cap02_teoria}

\subsection{Variación cuadrática}

La siguiente definición ha sido tomada de \cite{fuchsInferenceDiffusionProcesses2013}.

\begin{definition}
    Sea $X$ un proceso de Itô observado continuamente en el intervalo $[s,t]$, entonces
    la \textbf{variación cuadrática} del proceso en $[s,t]$ se define como
    \begin{equation}\label{eq:QV}
        [X,X]_{s,t} = \lim\limits_{n\to 0}\sum_{k=1}^{2^n}\left(X_{t_{k}^{(n)}}-X_{t_{k-1}^{(n)}}\right)
        \left(X_{t_{k}^{(n)}}-X_{t_{k-1}^{(n)}}\right)'=\int_s^t g(X_\tau,\theta)g(X_\tau,\theta)'d\tau,
    \end{equation}
    donde $t_k^{(n)}=s+k2^{-n}(t-s)$ for $k=0,...,2^n$, y la segunda igualdad es en probabilidad y 
    casi seguramente.
\end{definition}


\section{Solución numérica de EDEs}\label{sec:metodos_numericos}
\textcite{rackauckasDifferentialEquationsjlPerformantFeatureRich2017a}
Dado que la solución analítica de una ecuación diferencial estocástica 
es, en general, imposible de calcular, se tiene que recurrir a 
métodos  numéricos que aproximen esa solución. Hay una amplia 
variedad de métodos numéricos desarrollados para este fin, sin 
embargo se optó por usar un método que es fácil de implementar
si el proceso de difusión es multivariado, el método de Euler-Maruyama.
Otros métodos existen y, en general, tienen un mejor desempeño 
pero su implementación implica un mayor esfuerzo. Entre estos 
se encuentran los métodos de Milstein, Runge-Kutta.

\subsection{Convergencia y consistencia}

\begin{definition}
    Una aproximación a tiempo discreto $Y^\delta$, donde $\delta$ es el 
    tamaño de paso máximo, tiene un nivel de convergencia fuerte $\gamma$
    al proceso $X$ si existe una constante positiva $K$, que no depende de 
    $\delta$, y un $\delta_0>0$ tal que 
    $$\mathbb E\left(|X_t - Y_t^\delta|\right)\leq K\delta^\gamma$$
    para todo $t\geq 0$ y toda $\delta \in (0,\delta_0)$.
\end{definition}

\subsection{Método de Euler-Maruyama}
Si quiere obtenerse 
una aproximación de la solución a la ecuación tal 
en el intervalo $[t_0,T]$ tal que $X_0 = x_0$. Considérese la partición 
$t_0=\tau_0<\tau_1<...<\tau_n=T$ y defínase $\Delta_i=\tau_{i+1}-\tau_i$
para $i=0,...,n-1$. El proceso estocástico continuo $Y=\{Y_t\}_{t\geq t_0}^T$ 
se define, generalmente, como la interpolación lineal del conjunto de puntos 
$\{(\tau_i, Y_{\tau_i}):i=0,...,n)\}$ donde 
$Y_{\tau_0}=x_0$ y el resto satisfacen la relación iterativa
\begin{equation}\label{eq:EM}
Y_{\tau_{i+1}}=Y_{\tau_{i}}+f(X_{\tau_i},\tau_i)\Delta_i 
+ g(Y_{\tau_i},\tau_i)\Delta_i(W_{\tau_{i+1}}-W_{\tau_i})   
\end{equation}
para  $i=0,...,n-1$. En general, se toma una partición con puntos
equidistantes, es decir, $\Delta_i = \Delta = \frac{T-t_0}{n}$ y $\tau_i = \tau_0+i\Delta$. 
Para  simular este proceso, se usa la normalidad de los incrementos 
del movimiento Browniano (ver algoritmo \ref{alg:EM}). La paquetería 
\texttt{DifferentialEquations.jl} permite resolver una ecuación estocástica
con este esquema especificnado \texttt{EM()} como método y
un valor para $\Delta$ por medio del parámetro \texttt{dt}, siendo 
$\Delta=\frac{T-t_0}{n}$, para algún $n\in \mathbb N_+$, de modo que 
el valor en la última posición del vector arrojado por el algoritmo 
corresponda a $Y_{T}$.

\begin{algorithm}
\caption{Esquema de Euler-Maruyama}
\label{alg:EM}
\begin{algorithmic}[1]
\vspace{0.2cm}
\Require \parbox[t]{13cm}{
    Función de deriva $f$, función de difusión $g$, tamaño de paso 
    $\Delta$, lapso de tiempo $(t_0,T)$, valor inicial $x_0$
}
\vspace{0.2cm}
\Ensure 
\parbox[t]{13cm}{
    Aproximación de la solución a $dX_t=f(X_t,t)dt+g(X_t,t)W_t$ sujeta a 
    $X_{t_0}=x_0$ evaluada en 
    $t\in\{t_i=t_0 + i\Delta:\ t_i\leq T,\ i\in \mathbb N_+\}$
}
\vspace{0.2cm}
\State $d \leftarrow \dim(x_0)$
\State $n\leftarrow \lfloor(T-t_0)/\Delta\rfloor$
\State $Y \leftarrow [n+1,d]$
\State $Y[1,:]\leftarrow x_0$
\For{$i\in 1,...,n$}
    \State $\varepsilon \sim N_d(0, I_d)$
    \State $t\leftarrow t_0+i\Delta$
    \State $Y[i+1,:]\leftarrow f(Y[i,:],t)\cdot \Delta+g(Y[i,:],t)\cdot \Delta\cdot  \varepsilon$
\EndFor
\end{algorithmic}
\end{algorithm}

El método de Euler-Maruyama tiene un orden de convergencia de 
$\gamma = \frac{1}{2}$, mientras que los métodos de Milstein y 
Runge-Kutta tienen tal y tal, respectivamente 
\cite{fuchsInferenceDiffusionProcesses2013}.

\subsection{Esquema de Milstein}

De forma análoga que en el método de Euler-Maruyama, para construir la aproximación
de Milstein debe considerarse una partición $t_0=\tau_0<\tau_1<...<\tau_n=T$
equidistante, de forma que $\Delta_i=\tau_{i+1}-\tau_{i}$ 
sea una constante $\Delta$ para todo $i=0,...,n-1$. 
Entonces la componente $k$-ésima del proceso aproximación cumple la 
relación iterativa siguiente 
\begin{equation}\label{eq:Milstein}
    \begin{split}
        Y_{\tau_{i+1}}^{(k)}=&Y_{\tau_i}^{(k)}+f_k(Y_{\tau_{i}},\tau_i)\Delta
+ \sum_{j=1}^m g_{kj}(Y_{\tau_i},\tau_i)\Delta W_i^{(j)}\\
&+\sum_{j,l=1}^m\sum_{r=1}^dg_{rj}(Y_{\tau_i},\tau_i)
\left(\frac{\partial g_{kl}}{\delta x^{(r)}}(Y_{\tau_i},\tau_i) 
\right)\int\limits^{\tau_{i+1}}_{\tau_i} \int\limits^s_{\tau_i}dW_u^{(j)}dW_s^{(l)}
    \end{split}
\end{equation}
para $i=0,...,n-1$ y donde $\Delta W_{i}=W_{\tau_{i+1}}-W_{\tau_i}$. Para una 
dimensión, la implementación del método no tiene mayor complicación, 
sin embargo, en el caso multivariado el cálculo de las derivadas parciales 
de $g$ y de la doble integral representa un problema que puede atacarse 
desde distintos enfoques y que, en ciertas condiciones, se simplifica 
\cite{kloedenNumericalSolutionStochastic1992}. 
%Por ejemplo, cuando el coeficiente de difusión es diagonal 


\subsection{Esquema de Runge-Kutta-Milstein}
% pp. 373 kloeden 
Tal como en el caso determinista, el esquema de Milstein (eq. \ref{eq:Milstein})
puede ser modificado con el fin de evitar la evaluación de las derivadas del coeficiente
de difusión. Usando diferencias finitas se llega al esquema de Runge-Kutta-Milstein, 
referenciado en \cite{kloedenNumericalSolutionStochastic1992} como esquema fuerte de orden
1.0 explícito. Considerando de nuevo una partición equidistante, la $k$-ésima 
componente de esta aproximación cumple la siguiente relación iterativa
% Explicit Order 1.0 Strong Schemes
\begin{equation}\label{eq:Runge-Kutta-Milstein}
    \begin{split}
    Y_{\tau_{i+1}}^{(k)}=&Y_{\tau_i}^{(k)} + f_k(Y_{\tau_i},\tau_i)\Delta + 
    \sum_{j=1}^m g_{kj}(Y_{\tau_i},\tau_i)\Delta W_i^{(j)}\\ &
    + \frac{1}{\sqrt{\Delta}}\sum_{j,l=1}^m \left(g_{k,l}(\tilde Y_{j,\tau_i}, \tau_i)- g_{k,l}(Y_{\tau_i},\tau_i) 
    \right)\int\limits^{\tau_{i+1}}_{\tau_i} \int\limits^s_{\tau_i}dW_u^{(j)}dW_s^{(l)}\\
    \tilde Y_{j,\tau_i}=& Y_{\tau_i}+f(Y_{\tau_i},\tau_i)\Delta + g_{:,j}(Y_{\tau_i},\tau_i)\sqrt{\Delta}
    \end{split}
\end{equation}
% sigma_{:,j} j-ésima columna de sigma
para $i=0,...,n-1$ y $j=1,...,m$ y donde $\Delta W_{i}=W_{\tau_{i+1}}-W_{\tau_i}$ y $g_{:,j}$ es la $j$-ésima
columna de $g$. Esta aproximación es fuertemente consistente de orden $\gamma = 1.0$, si los
coeficientes son dos veces diferenciables y tales derivadas son uniformemente acotadas \cite{fuchsInferenceDiffusionProcesses2013}.
Este esquema está implementado en \texttt{DifferentialEquations.jl} como \texttt{RKMil} y requiere de
un tamaño de paso fijo.


\chapter{Aplicaciones a modelos epidemiológicos compartamentales}
\label{cap:04}
\section{Teoria}\label{sec:cap02_teoria}

\subsection{Variación cuadrática}

La siguiente definición ha sido tomada de \cite{fuchsInferenceDiffusionProcesses2013}.

\begin{definition}
    Sea $X$ un proceso de Itô observado continuamente en el intervalo $[s,t]$, entonces
    la \textbf{variación cuadrática} del proceso en $[s,t]$ se define como
    \begin{equation}\label{eq:QV}
        [X,X]_{s,t} = \lim\limits_{n\to 0}\sum_{k=1}^{2^n}\left(X_{t_{k}^{(n)}}-X_{t_{k-1}^{(n)}}\right)
        \left(X_{t_{k}^{(n)}}-X_{t_{k-1}^{(n)}}\right)'=\int_s^t g(X_\tau,\theta)g(X_\tau,\theta)'d\tau,
    \end{equation}
    donde $t_k^{(n)}=s+k2^{-n}(t-s)$ for $k=0,...,2^n$, y la segunda igualdad es en probabilidad y 
    casi seguramente.
\end{definition}


\section{Solución numérica de EDEs}\label{sec:metodos_numericos}
\textcite{rackauckasDifferentialEquationsjlPerformantFeatureRich2017a}
Dado que la solución analítica de una ecuación diferencial estocástica 
es, en general, imposible de calcular, se tiene que recurrir a 
métodos  numéricos que aproximen esa solución. Hay una amplia 
variedad de métodos numéricos desarrollados para este fin, sin 
embargo se optó por usar un método que es fácil de implementar
si el proceso de difusión es multivariado, el método de Euler-Maruyama.
Otros métodos existen y, en general, tienen un mejor desempeño 
pero su implementación implica un mayor esfuerzo. Entre estos 
se encuentran los métodos de Milstein, Runge-Kutta.

\subsection{Convergencia y consistencia}

\begin{definition}
    Una aproximación a tiempo discreto $Y^\delta$, donde $\delta$ es el 
    tamaño de paso máximo, tiene un nivel de convergencia fuerte $\gamma$
    al proceso $X$ si existe una constante positiva $K$, que no depende de 
    $\delta$, y un $\delta_0>0$ tal que 
    $$\mathbb E\left(|X_t - Y_t^\delta|\right)\leq K\delta^\gamma$$
    para todo $t\geq 0$ y toda $\delta \in (0,\delta_0)$.
\end{definition}

\subsection{Método de Euler-Maruyama}
Si quiere obtenerse 
una aproximación de la solución a la ecuación tal 
en el intervalo $[t_0,T]$ tal que $X_0 = x_0$. Considérese la partición 
$t_0=\tau_0<\tau_1<...<\tau_n=T$ y defínase $\Delta_i=\tau_{i+1}-\tau_i$
para $i=0,...,n-1$. El proceso estocástico continuo $Y=\{Y_t\}_{t\geq t_0}^T$ 
se define, generalmente, como la interpolación lineal del conjunto de puntos 
$\{(\tau_i, Y_{\tau_i}):i=0,...,n)\}$ donde 
$Y_{\tau_0}=x_0$ y el resto satisfacen la relación iterativa
\begin{equation}\label{eq:EM}
Y_{\tau_{i+1}}=Y_{\tau_{i}}+f(X_{\tau_i},\tau_i)\Delta_i 
+ g(Y_{\tau_i},\tau_i)\Delta_i(W_{\tau_{i+1}}-W_{\tau_i})   
\end{equation}
para  $i=0,...,n-1$. En general, se toma una partición con puntos
equidistantes, es decir, $\Delta_i = \Delta = \frac{T-t_0}{n}$ y $\tau_i = \tau_0+i\Delta$. 
Para  simular este proceso, se usa la normalidad de los incrementos 
del movimiento Browniano (ver algoritmo \ref{alg:EM}). La paquetería 
\texttt{DifferentialEquations.jl} permite resolver una ecuación estocástica
con este esquema especificnado \texttt{EM()} como método y
un valor para $\Delta$ por medio del parámetro \texttt{dt}, siendo 
$\Delta=\frac{T-t_0}{n}$, para algún $n\in \mathbb N_+$, de modo que 
el valor en la última posición del vector arrojado por el algoritmo 
corresponda a $Y_{T}$.

\begin{algorithm}
\caption{Esquema de Euler-Maruyama}
\label{alg:EM}
\begin{algorithmic}[1]
\vspace{0.2cm}
\Require \parbox[t]{13cm}{
    Función de deriva $f$, función de difusión $g$, tamaño de paso 
    $\Delta$, lapso de tiempo $(t_0,T)$, valor inicial $x_0$
}
\vspace{0.2cm}
\Ensure 
\parbox[t]{13cm}{
    Aproximación de la solución a $dX_t=f(X_t,t)dt+g(X_t,t)W_t$ sujeta a 
    $X_{t_0}=x_0$ evaluada en 
    $t\in\{t_i=t_0 + i\Delta:\ t_i\leq T,\ i\in \mathbb N_+\}$
}
\vspace{0.2cm}
\State $d \leftarrow \dim(x_0)$
\State $n\leftarrow \lfloor(T-t_0)/\Delta\rfloor$
\State $Y \leftarrow [n+1,d]$
\State $Y[1,:]\leftarrow x_0$
\For{$i\in 1,...,n$}
    \State $\varepsilon \sim N_d(0, I_d)$
    \State $t\leftarrow t_0+i\Delta$
    \State $Y[i+1,:]\leftarrow f(Y[i,:],t)\cdot \Delta+g(Y[i,:],t)\cdot \Delta\cdot  \varepsilon$
\EndFor
\end{algorithmic}
\end{algorithm}

El método de Euler-Maruyama tiene un orden de convergencia de 
$\gamma = \frac{1}{2}$, mientras que los métodos de Milstein y 
Runge-Kutta tienen tal y tal, respectivamente 
\cite{fuchsInferenceDiffusionProcesses2013}.

\subsection{Esquema de Milstein}

De forma análoga que en el método de Euler-Maruyama, para construir la aproximación
de Milstein debe considerarse una partición $t_0=\tau_0<\tau_1<...<\tau_n=T$
equidistante, de forma que $\Delta_i=\tau_{i+1}-\tau_{i}$ 
sea una constante $\Delta$ para todo $i=0,...,n-1$. 
Entonces la componente $k$-ésima del proceso aproximación cumple la 
relación iterativa siguiente 
\begin{equation}\label{eq:Milstein}
    \begin{split}
        Y_{\tau_{i+1}}^{(k)}=&Y_{\tau_i}^{(k)}+f_k(Y_{\tau_{i}},\tau_i)\Delta
+ \sum_{j=1}^m g_{kj}(Y_{\tau_i},\tau_i)\Delta W_i^{(j)}\\
&+\sum_{j,l=1}^m\sum_{r=1}^dg_{rj}(Y_{\tau_i},\tau_i)
\left(\frac{\partial g_{kl}}{\delta x^{(r)}}(Y_{\tau_i},\tau_i) 
\right)\int\limits^{\tau_{i+1}}_{\tau_i} \int\limits^s_{\tau_i}dW_u^{(j)}dW_s^{(l)}
    \end{split}
\end{equation}
para $i=0,...,n-1$ y donde $\Delta W_{i}=W_{\tau_{i+1}}-W_{\tau_i}$. Para una 
dimensión, la implementación del método no tiene mayor complicación, 
sin embargo, en el caso multivariado el cálculo de las derivadas parciales 
de $g$ y de la doble integral representa un problema que puede atacarse 
desde distintos enfoques y que, en ciertas condiciones, se simplifica 
\cite{kloedenNumericalSolutionStochastic1992}. 
%Por ejemplo, cuando el coeficiente de difusión es diagonal 


\subsection{Esquema de Runge-Kutta-Milstein}
% pp. 373 kloeden 
Tal como en el caso determinista, el esquema de Milstein (eq. \ref{eq:Milstein})
puede ser modificado con el fin de evitar la evaluación de las derivadas del coeficiente
de difusión. Usando diferencias finitas se llega al esquema de Runge-Kutta-Milstein, 
referenciado en \cite{kloedenNumericalSolutionStochastic1992} como esquema fuerte de orden
1.0 explícito. Considerando de nuevo una partición equidistante, la $k$-ésima 
componente de esta aproximación cumple la siguiente relación iterativa
% Explicit Order 1.0 Strong Schemes
\begin{equation}\label{eq:Runge-Kutta-Milstein}
    \begin{split}
    Y_{\tau_{i+1}}^{(k)}=&Y_{\tau_i}^{(k)} + f_k(Y_{\tau_i},\tau_i)\Delta + 
    \sum_{j=1}^m g_{kj}(Y_{\tau_i},\tau_i)\Delta W_i^{(j)}\\ &
    + \frac{1}{\sqrt{\Delta}}\sum_{j,l=1}^m \left(g_{k,l}(\tilde Y_{j,\tau_i}, \tau_i)- g_{k,l}(Y_{\tau_i},\tau_i) 
    \right)\int\limits^{\tau_{i+1}}_{\tau_i} \int\limits^s_{\tau_i}dW_u^{(j)}dW_s^{(l)}\\
    \tilde Y_{j,\tau_i}=& Y_{\tau_i}+f(Y_{\tau_i},\tau_i)\Delta + g_{:,j}(Y_{\tau_i},\tau_i)\sqrt{\Delta}
    \end{split}
\end{equation}
% sigma_{:,j} j-ésima columna de sigma
para $i=0,...,n-1$ y $j=1,...,m$ y donde $\Delta W_{i}=W_{\tau_{i+1}}-W_{\tau_i}$ y $g_{:,j}$ es la $j$-ésima
columna de $g$. Esta aproximación es fuertemente consistente de orden $\gamma = 1.0$, si los
coeficientes son dos veces diferenciables y tales derivadas son uniformemente acotadas \cite{fuchsInferenceDiffusionProcesses2013}.
Este esquema está implementado en \texttt{DifferentialEquations.jl} como \texttt{RKMil} y requiere de
un tamaño de paso fijo.


\chapter{Conclusiones}
\label{cap:05}
%\cite{aliStochasticModelingInfluenza2024}
%\subfile{capitulos/conclusiones}
% \bibliographystyle{ieeetr}    % o 'apalike', 'ieeetr', 'alpha', etc.
% \bibliography{references.bib} 
\printbibliography
\end{document}